% ==========================================
% BAB II STUDI LITERATUR
% ==========================================
\chapter{STUDI LITERATUR}
\label{chap:studi-literatur}

\section{Malware dan Spyware}
\textit{Malware} adalah perangkat lunak berbahaya yang dirancang untuk merusak, mengakses tanpa izin, atau mengganggu sistem komputer. Salah satu tipe \textit{malware} yang khusus adalah \textit{spyware}, yaitu program yang bertujuan untuk memantau dan mengumpulkan informasi sensitif pengguna secara diam-diam tanpa sepengetahuan mereka. \textit{Spyware} dapat melakukan berbagai aktivitas seperti menangkap penekanan tombol (keylogging), memantau aktivitas sistem, dan mencuri data pribadi.

Perkembangan \textit{spyware} sangat cepat seiring dengan kemajuan teknologi serta alat bantu yang tersedia bagi pembuat \textit{malware}. Misalnya, penggunaan alat seperti \textit{MSF Venom} memungkinkan pembuatan \textit{payload spyware} yang dapat dikustomisasi untuk menjalankan aktivitas berbahaya seperti pencurian data dan penyadapan jaringan. Deteksi \textit{spyware} memerlukan pendekatan multifaset yang mengombinasikan analisis perilaku sistem, pengecekan tanda tangan berkas, serta pemantauan pola lalu lintas jaringan.

Menurut \textit{Reddyvari et al.}~\cite{reddyvari2024}, ``\textit{Spyware} adalah jenis \textit{malware} yang dirancang untuk memantau dan mengumpulkan informasi sensitif tanpa izin, menimbulkan ancaman serius terhadap privasi dan keamanan.'' Mereka juga menyebutkan bahwa alat seperti \textit{YARA} dapat digunakan untuk membuat aturan deteksi yang spesifik terhadap tanda-tanda dan pola khas \textit{spyware}, sehingga membantu dalam identifikasi dan mitigasi ancaman tersebut.

Penggunaan kombinasi teknologi seperti modul \textit{Python} untuk menganalisis perilaku \textit{malware} dan \textit{YARA rules} untuk identifikasi pola terbukti efektif dalam mendeteksi \textit{spyware} baru yang sulit dikenali oleh antivirus tradisional. Dengan demikian, pemahaman mendalam tentang karakteristik \textit{malware} dan \textit{spyware} sangat penting sebagai dasar bagi pengembangan metode pencegahan serta deteksi keamanan siber yang lebih maju.

\section{Malware Fileless}
\textit{Malware fileless} merupakan varian \textit{malware} yang tidak menggunakan berkas eksekusi tradisional pada sistem target, melainkan beroperasi langsung di memori (\textit{RAM}) atau memanfaatkan komponen sistem yang sudah ada seperti \textit{PowerShell} dan \textit{Windows Management Instrumentation} (WMI). Teknik ini membuat \textit{malware fileless} sulit dideteksi oleh solusi keamanan konvensional yang biasanya mengandalkan pemindaian berkas di media penyimpanan.

\textit{Malware fileless} umumnya mengeksploitasi kerentanan dalam aplikasi yang sah dan menggunakan skrip atau perintah yang berjalan di memori untuk menjalankan aksinya sehingga kehadirannya hampir tidak meninggalkan jejak pada sistem berkas. Salah satu contoh serangan terkenal dengan metode ini adalah infiltrasi yang dilakukan melalui skrip \textit{PowerShell} yang dieksekusi tanpa menyimpan berkas berbahaya di media penyimpanan.

Menurut \textit{Sudhakar dan Kumar}~\cite{sudhakar2020}, \textit{malware fileless} tidak pernah menulis berkas berbahaya ke media penyimpanan, melainkan menyuntikkan kode ke dalam memori dan menggunakan alat yang telah dipercaya oleh sistem operasi untuk tetap tersembunyi serta mempertahankan persistensi. Mereka juga menjelaskan bahwa \textit{malware fileless} memanfaatkan fitur Windows seperti WMI dan \textit{PowerShell} untuk \textit{reconnaissance}, eksekusi kode, pergerakan lateral, \textit{persistence}, serta pencurian data. Hal ini membuat deteksi \textit{malware} jenis ini menjadi tantangan besar bagi sistem keamanan yang ada karena serangan ini sangat \textit{stealthy} dan sulit dikenali menggunakan tanda tangan tradisional.

Penelitian \textit{Elghaly}~\cite{elghaly2024} menemukan bahwa \textit{malware fileless} berbasis \textit{PowerShell} semakin marak digunakan oleh kelompok ancaman terorganisir (\textit{APT}) dan penjahat siber karena kemampuannya menjadikan dirinya ``tak terlihat'' oleh sistem pertahanan seperti \textit{Endpoint Detection and Response} (EDR) dan antivirus modern. Studi ini juga menunjukkan bahwa teknik \textit{obfuscation PowerShell} semakin mempersulit proses deteksi dan analisis oleh solusi keamanan yang ada.

Secara umum, \textit{malware fileless} mengubah lanskap keamanan siber dengan mengadopsi metode serangan yang tidak bergantung pada berkas tradisional untuk mengeksekusi \textit{malware}. Kondisi ini menuntut pengembangan teknologi deteksi baru seperti analisis perilaku dan pemantauan memori secara \textit{real-time} untuk meningkatkan efektivitas mitigasi serta perlindungan terhadap ancaman tersebut.

\section{Dropper}
\textit{Dropper} adalah jenis \textit{malware} yang berfungsi sebagai perantara untuk menempatkan \textit{malware} utama ke dalam sistem target secara diam-diam. \textit{Dropper} biasanya menyamar sebagai berkas atau program yang tampak sah agar dapat dijalankan oleh pengguna tanpa kecurigaan. Setelah dieksekusi, \textit{dropper} akan mengekstrak dan memasang \textit{malware} tambahan, seperti \textit{spyware}, \textit{ransomware}, atau \textit{backdoor}, ke dalam sistem korban.

Fungsi utama \textit{dropper} adalah membawa dan mengantarkan \textit{payload malware} sehingga infeksi dapat berlangsung secara efektif. \textit{Dropper} dapat membawa \textit{malware} tersembunyi di dalam dirinya atau mengunduh \textit{malware} dari server eksternal setelah berhasil dieksekusi. Ada dua tipe \textit{dropper}, yaitu \textit{persistent dropper} yang mampu bertahan lama dan melakukan instal ulang \textit{malware}, serta \textit{non-persistent dropper} yang hanya bekerja satu kali.

Penelitian \textit{Chatzoglou et al.}~\cite{chatzoglou2023} menjelaskan bahwa berbagai teknik evasi sering digunakan oleh \textit{dropper} untuk menghindari deteksi antivirus, seperti enkripsi, \textit{obfuscation}, dan pembelahan kode \textit{malware} menjadi beberapa bagian. Teknik injeksi proses juga umum digunakan, di mana \textit{dropper} menyisipkan kode berbahaya ke dalam proses yang sah agar dapat berjalan secara \textit{stealth} dan sulit terdeteksi oleh sistem keamanan.

\textit{Dropper} menjadi komponen krusial dalam siklus infeksi \textit{malware} karena menentukan keberhasilan pengantaran dan eksekusi \textit{payload malware} di sistem target. Oleh karena itu, kemampuan \textit{dropper} untuk menyamar dan melakukan evasi deteksi menjadi aspek yang sangat penting dalam pengembangan maupun penanggulangan \textit{malware}.

\section{Teknik Evasi Malware}
Teknik evasi \textit{malware} dirancang untuk menghindari deteksi dan analisis oleh perangkat keamanan seperti antivirus dan sistem deteksi intrusi. Beberapa teknik utama meliputi \textit{code obfuscation}, penggunaan \textit{packer} untuk menyamarkan berkas eksekusi, serta injeksi proses di mana \textit{malware} menyisipkan kode berbahaya ke dalam proses yang sah agar sulit dideteksi. Teknik persistensi juga umum dilakukan dengan memanfaatkan \textit{registry} atau \textit{task scheduler} sehingga \textit{malware} tetap aktif meskipun sistem direstart. \textit{Malware} modern kerap melakukan pengecekan terhadap lingkungan eksekusi, seperti \textit{sandbox} atau \textit{virtual machine}, untuk menyesuaikan perilaku dan menghindari analisis otomatis.

\section{Anti Malware dan Sistem Deteksi}
\textit{Anti-malware} merupakan perangkat lunak yang dirancang untuk mendeteksi, mencegah, dan menghapus \textit{malware}, termasuk \textit{spyware}, dari sistem komputer. Sistem deteksi \textit{malware} tradisional biasanya menggunakan metode berbasis tanda tangan (\textit{signature-based}) yang mengidentifikasi \textit{malware} berdasarkan ciri-ciri spesifik yang telah diketahui. Namun, metode ini kurang efektif untuk mendeteksi \textit{malware} baru atau \textit{malware} yang menggunakan teknik evasi canggih.

Untuk mengatasi keterbatasan tersebut, teknologi deteksi saat ini mulai mengadopsi metode \textit{heuristic} dan \textit{behavior-based}, yang memantau perilaku mencurigakan di sistem, serta menggunakan \textit{machine learning} dan \textit{big data} untuk mengidentifikasi pola serangan baru. Dengan analisis perilaku, sistem dapat mengenali aktivitas yang tidak biasa meskipun tanda tangan \textit{malware} belum tersedia.

Meskipun demikian, \textit{malware} modern, terutama \textit{spyware fileless}, semakin sulit dideteksi karena kemampuannya beroperasi di memori dan memanfaatkan fitur sistem yang sah. Oleh karenanya, pengujian kapabilitas \textit{anti-malware} dalam skenario dunia nyata sangat penting untuk mengetahui sejauh mana perlindungan yang diberikan sekaligus mendorong pengembangan teknologi keamanan lebih lanjut.

\section{Studi Kasus Spyware Modern}
\textit{Spyware} modern berkembang dengan kemampuan yang sangat canggih sehingga memungkinkan pengintaian dan pencurian data secara tersembunyi dari perangkat target. Salah satu contohnya adalah \textit{spyware Pegasus} yang dikembangkan oleh \textit{NSO Group}, yang mampu mengeksploitasi kerentanan \textit{zero-click} untuk menginfeksi \textit{smartphone} tanpa interaksi pengguna. \textit{Spyware} ini dapat mengakses pesan, panggilan, lokasi, serta data sensitif lainnya secara \textit{real-time} tanpa terdeteksi oleh pemilik perangkat.

Analisis terhadap \textit{Pegasus} menyoroti risiko besar terhadap privasi dan keamanan digital, sekaligus menunjukkan bagaimana \textit{spyware} modern memanfaatkan teknik penyebaran yang sangat tersembunyi dan sulit diatasi dengan metode pertahanan tradisional. Penggunaan teknologi canggih tersebut menuntut pengembangan kebijakan dan teknologi baru untuk melindungi data serta hak privasi pengguna dari ancaman \textit{spyware} yang semakin maju.

Menurut \textit{Kareem}~\cite{kareem2024}, ``\textit{Pegasus} menunjukkan bahwa \textit{spyware} modern berpotensi menjadi alat pengawasan yang sangat kuat dan berbahaya, sehingga perlindungan dan regulasi yang ketat sangat diperlukan untuk mencegah penyalahgunaan teknologi tersebut.'' Dengan demikian, studi kasus \textit{spyware} modern seperti \textit{Pegasus} memberikan gambaran nyata tentang kompleksitas ancaman yang dihadapi keamanan digital saat ini dan pentingnya pengujian sistem keamanan secara menyeluruh terhadap ancaman \textit{spyware} baru.
