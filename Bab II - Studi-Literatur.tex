% ==========================================
% BAB II TINJAUAN PUSTAKA
% ==========================================
\chapter{TINJAUAN PUSTAKA}
\label{chap:tinjauan-pustaka}

\section{Malware}
\textit{Malware}, atau \textit{malicious software}, merupakan perangkat lunak yang dirancang secara sengaja untuk melakukan aktivitas merusak, mengganggu, atau mengakses sistem tanpa izin pengguna. Istilah ini mencakup berbagai jenis ancaman siber seperti \textit{virus}, \textit{worm}, \textit{trojan horse}, \textit{ransomware}, dan \textit{spyware} yang masing-masing memiliki mekanisme infeksi dan tujuan spesifik. \textit{Malware} modern tidak lagi hanya bertujuan untuk merusak data atau sistem, tetapi semakin berorientasi pada pencurian informasi sensitif, penciptaan \textit{botnet} untuk serangan DDoS, atau bahkan sebagai vektor untuk serangan \textit{ransomware} yang menuntut pembayaran tebusan. Evolusi \textit{malware} ini menunjukkan pergeseran paradigma dari ancaman yang mudah dideteksi melalui tanda tangan digital (\textit{signature-based}) menuju varian yang menggunakan teknik \textit{obfuscation} dan \textit{polymorphism} untuk menghindari deteksi sistem keamanan konvensional \parencite{sudhakar2020}.

Dalam konteks keamanan siber saat ini, \textit{malware} telah menjadi alat utama dalam \textit{cyber kill chain} yang sistematis, dimulai dari \textit{initial access} hingga \textit{command and control}. Kompleksitas \textit{malware} meningkat seiring dengan adopsi teknik \textit{fileless execution} dan \textit{living off the land binaries} (LOLBins), di mana \textit{malware} memanfaatkan komponen sistem operasi yang sah seperti \textit{PowerShell} atau \textit{WMI} untuk eksekusi tanpa meninggalkan jejak file di disk. Hal ini menciptakan tantangan fundamental bagi sistem deteksi tradisional yang bergantung pada analisis \textit{file-based}, karena \textit{malware} kini beroperasi sepenuhnya di memori atau melalui proses yang tampak \textit{legitimate}. Penelitian menunjukkan bahwa serangan \textit{malware} global mencapai 5,5 miliar insiden pada 2022, dengan peningkatan 2\% dari tahun sebelumnya, menandakan eskalasi ancaman yang berkelanjutan \parencite{chatzoglou2023}.

Relevansi \textit{malware} dengan penelitian ini terletak pada peran sentralnya sebagai fondasi bagi pengembangan \textit{dropper spyware} mode \textit{stealth}. \textit{Dropper}, sebagai salah satu varian \textit{malware}, berfungsi sebagai \textit{delivery mechanism} yang mengantarkan \textit{payload spyware} ke sistem target sambil menghindari deteksi awal. Kelemahan sistem \textit{anti-malware} konvensional dalam mendeteksi \textit{malware evasif} menjadi dasar utama penelitian ini, yang bertujuan menguji kapabilitas \textit{anti-spyware} terhadap \textit{dropper} yang mengintegrasikan teknik evasi modern. Tanpa pemahaman mendalam tentang evolusi \textit{malware} dari ancaman tradisional ke varian \textit{stealth} seperti \textit{fileless} dan \textit{dropper}-\textit{based}, pengembangan solusi pertahanan yang efektif akan terus tertinggal dari inovasi penyerang. Oleh karena itu, analisis \textit{malware} tidak hanya memberikan konteks teoritis, tetapi juga landasan empiris untuk merancang prototipe \textit{dropper} yang realistis dalam lingkungan pengujian terkontrol \parencite{koutsokostas2021}.

Dalam kerangka MITRE ATT\&CK, \textit{malware} mencakup berbagai taktik seperti \textit{Execution}, \textit{Persistence}, \textit{Privilege Escalation}, dan \textit{Defense Evasion} yang saling terkait membentuk rantai serangan yang kohesif. Pemahaman ini krusial untuk mengidentifikasi tahap mana yang paling rentan terhadap deteksi, khususnya pada fase \textit{initial access} yang menjadi fokus utama \textit{dropper spyware}. Studi empiris menunjukkan bahwa 95\% CEO menganggap keamanan siber sebagai ancaman utama pertumbuhan bisnis pada 2021, naik dari 61\% tahun sebelumnya, yang menegaskan urgensi penelitian terhadap kemampuan deteksi \textit{malware} modern \parencite{ainslie2023}. Dengan demikian, tinjauan \textit{malware} ini menjadi titik awal untuk memahami mengapa evaluasi \textit{anti-spyware} terhadap \textit{dropper stealth} menjadi imperatif dalam lanskap ancaman siber kontemporer.

\section{Spyware}
\textit{Spyware} merupakan jenis \textit{malware} yang secara spesifik dirancang untuk memantau, mengumpulkan, dan mengirimkan data sensitif dari sistem target tanpa sepengetahuan atau persetujuan pengguna. Berbeda dengan \textit{ransomware} yang fokus pada enkripsi data untuk pemerasan atau \textit{trojan} yang bertujuan akses \textit{backdoor}, \textit{spyware} beroperasi secara diam-diam dengan tujuan pencurian informasi seperti kredensial \textit{login}, riwayat aktivitas, lokasi geografis, hingga percakapan pribadi melalui mikrofon dan kamera. Karakteristik utama \textit{spyware} adalah kemampuan \textit{stealth} dan \textit{persistensi}, di mana ia mampu bertahan di sistem selama berbulan-bulan bahkan bertahun-tahun tanpa terdeteksi oleh pengguna atau sistem keamanan standar \parencite{reddyvari2024}.

Mekanisme infeksi \textit{spyware} modern semakin canggih, memanfaatkan \textit{zero-click exploits} yang tidak memerlukan interaksi pengguna, seperti kerentanan dalam aplikasi pesan instan atau peramban. Studi kasus \textit{Pegasus} dari NSO Group menunjukkan bagaimana \textit{spyware} tingkat negara dapat menginfeksi \textit{iPhone} dan \textit{Android} hanya melalui panggilan \textit{WhatsApp} yang dibatalkan, memungkinkan akses penuh ke data perangkat tanpa meninggalkan jejak signifikan. \textit{Spyware} ini mampu mengekstrak pesan terenkripsi, merekam panggilan, dan melacak lokasi \textit{real-time}, menjadikannya alat pengawasan yang sangat efektif namun kontroversial secara etis dan hukum \parencite{kareem2024}. Dalam konteks \textit{cyber kill chain}, \textit{spyware} biasanya berada pada tahap \textit{Collection} dan \textit{Exfiltration}, tetapi ketergantungannya pada \textit{dropper} sebagai vektor \textit{initial access} menjadikan rantai infeksi ini rentan pada tahap awal pengiriman \textit{payload}.

Ancaman \textit{spyware} tidak hanya terbatas pada individu, tetapi juga mencakup infrastruktur kritis dan organisasi pemerintah. Laporan menunjukkan bahwa \textit{spyware} komersial seperti \textit{Pegasus} telah digunakan terhadap jurnalis, aktivis HAM, dan pejabat tinggi di 45 negara, menimbulkan implikasi serius terhadap privasi digital dan keamanan nasional. Yang lebih mengkhawatirkan, \textit{spyware} modern mengintegrasikan teknik evasi lanjutan seperti \textit{fileless execution} dan komunikasi \textit{C2} terenkripsi, membuatnya sulit dideteksi oleh solusi \textit{anti-malware} konvensional yang masih mengandalkan \textit{signature matching} \parencite{kareem2024,reddyvari2024}. Keterbatasan ini menjadi dasar krusial bagi penelitian ini, yang secara spesifik menguji kapabilitas \textit{anti-spyware} terhadap \textit{dropper spyware} mode \textit{stealth}, komponen pengantar yang menentukan keberhasilan infeksi \textit{spyware} secara keseluruhan.

Relevansi \textit{spyware} dengan penelitian ini terletak pada posisinya sebagai \textit{payload} utama yang akan diuji melalui prototipe \textit{dropper}. Tanpa \textit{dropper} yang efektif dalam mengantarkan dan mengaktifkan \textit{spyware} tanpa deteksi, seluruh rantai serangan menjadi tidak efektif. Oleh karena itu, evaluasi terhadap kemampuan \textit{anti-spyware} dalam mendeteksi tahap \textit{delivery mechanism dropper} menjadi sangat kritis, terutama mengingat studi menunjukkan bahwa deteksi \textit{spyware} masih bergantung pada analisis perilaku pasca-infeksi, bukan pencegahan pada tahap awal \parencite{reddyvari2024}. Penelitian ini mengisi celah tersebut dengan mengembangkan dan menguji \textit{dropper} yang mensimulasikan serangan \textit{spyware} realistis, memberikan wawasan empiris tentang kelemahan sistem pertahanan saat ini terhadap ancaman yang semakin \textit{stealth} dan \textit{targeted}.

\section{Dropper}
\textit{Dropper} merupakan jenis \textit{malware} spesialis yang berfungsi sebagai \textit{delivery mechanism} atau pengantar untuk \textit{malware} utama, khususnya \textit{spyware}, dengan tugas utama mendistribusikan, membuka kemasan (\textit{unpack}), dan mengeksekusi \textit{payload} berbahaya di sistem target tanpa terdeteksi. Berbeda dengan \textit{malware} lain yang langsung menjalankan fungsi destruktif, \textit{dropper} dirancang untuk beroperasi secara \textit{stealth} pada tahap awal \textit{cyber kill chain}, yaitu \textit{initial access} dan \textit{execution}, memastikan \textit{payload spyware} dapat diaktifkan dengan sukses sebelum sistem keamanan bereaksi. \textit{Dropper} sering kali berupa \textit{executable} kecil yang tampak \textit{benign}, yang kemudian mengekstrak dan menjalankan \textit{malware} utama dari memori atau \textit{stream} terenkripsi, menjadikannya komponen kritis yang menentukan efektivitas seluruh rantai serangan \parencite{chatzoglou2023,elghaly2024}.

Arsitektur \textit{dropper} modern mencakup beberapa modul inti: \textit{payload container} yang menyimpan \textit{spyware} terenkripsi, \textit{unpacker}/\textit{decryptor} untuk ekstraksi \textit{payload}, \textit{environment checker} untuk mendeteksi analisis (\textit{anti-VM}, \textit{anti-sandbox}), dan \textit{loader} untuk eksekusi \textit{fileless} melalui injeksi proses atau \textit{reflective DLL loading}. Alur eksekusinya biasanya dimulai dengan validasi lingkungan target, diikuti dekripsi \textit{payload}, pengecekan integritas, dan aktivasi \textit{spyware} melalui teknik seperti \textit{process hollowing} atau \textit{APC injection} yang memanfaatkan proses sistem sah. Dalam kerangka MITRE ATT\&CK, \textit{dropper} mendominasi taktik T1105 (\textit{Ingress Tool Transfer}), T1027 (\textit{Obfuscated Files or Information}), dan T1055 (\textit{Process Injection}), menjadikannya titik lemah potensial dalam rantai serangan jika sistem deteksi mampu mengidentifikasi perilakunya pada tahap awal \parencite{chatzoglou2023}.

Peran strategis \textit{dropper} dalam infeksi \textit{spyware} tidak dapat diremehkan, karena kegagalannya berarti seluruh operasi pengintaian gagal sebelum dimulai. Penelitian empiris menunjukkan bahwa \textit{dropper} dengan teknik evasi sederhana seperti enkripsi XOR dan \textit{junk data insertion} mampu melewati 50\% produk \textit{antivirus} komersial, sementara integrasi \textit{anti-analysis mechanisms} meningkatkan tingkat keberhasilan hingga 80\% terhadap \textit{scanner} \textit{multi-engine} seperti \textit{VirusTotal}. \textit{Dropper} juga berevolusi menjadi varian \textit{non-persistent} yang menghilang setelah \textit{payload} aktif, meninggalkan jejak minimal untuk analisis forensik \parencite{koutsokostas2021}. Konteks ini sangat relevan dengan penelitian ini, yang mengembangkan prototipe \textit{dropper spyware} mode \textit{stealth} menggunakan Rust untuk mensimulasikan serangan realistis dan mengukur kapabilitas deteksi \textit{anti-spyware} pada tahap \textit{delivery}, fase yang paling kritis namun jarang dievaluasi secara komprehensif.

Kelemahan utama dalam pemahaman \textit{dropper} saat ini adalah kurangnya evaluasi terhadap integrasi \textit{multiple evasion techniques} dalam satu artefak, di mana penelitian terpisah hanya menguji \textit{packing}, \textit{anti-VM}, atau \textit{fileless} secara individual. Penelitian ini mengatasi \textit{gap} tersebut dengan prototipe holistik yang menggabungkan teknik-teknik tersebut, memberikan bukti empiris tentang sejauh mana produk \textit{anti-spyware} mampu mendeteksi dan menangkal \textit{dropper} sebagai vektor infeksi \textit{spyware} modern. Dengan demikian, analisis \textit{dropper} tidak hanya memberikan landasan teoritis, tetapi juga metodologi pengujian yang dapat direplikasi untuk meningkatkan maturitas sistem pertahanan siber \parencite{elghaly2024}.

\section{Fileless Malware}
\textit{Fileless malware} merupakan evolusi canggih dari ancaman tradisional yang sepenuhnya menghindari penggunaan file eksekusi di disk, beroperasi langsung melalui memori sistem atau memanfaatkan \textit{living off the land binaries} (LOLBins) seperti \textit{PowerShell}, \textit{WMI}, atau \textit{registry} untuk eksekusi kode berbahaya. Berbeda dengan \textit{malware} konvensional yang meninggalkan artefak file yang dapat dipindai oleh \textit{antivirus}, \textit{fileless malware} menyamarkan aktivitasnya sebagai proses sistem sah, sehingga lolos dari deteksi \textit{signature-based} dan bahkan sebagian besar \textit{heuristic analysis}. Penelitian menunjukkan bahwa serangan \textit{fileless} meningkat secara eksponensial sejak 2017, dengan kasus seperti \textit{Poweliks} dan \textit{Kovter} yang menggunakan \textit{WMI} untuk \textit{persistensi} tanpa jejak disk, menjadikannya ancaman yang sangat sulit dilacak secara forensik \parencite{sudhakar2020}.

Mekanisme kerja \textit{fileless malware} biasanya melibatkan \textit{script-based execution} melalui \textit{interpreter} yang sudah ada di sistem target, seperti \textit{PowerShell} untuk \textit{reflective loading DLL} atau \textit{WMI event subscription} untuk \textit{persistensi}. Teknik ini memanfaatkan API Windows yang sah untuk injeksi kode ke proses seperti \textit{explorer.exe} atau \textit{svchost.exe}, menciptakan ilusi aktivitas normal sambil menjalankan \textit{payload spyware}. Studi kasus \textit{PowerShell fileless malware} menunjukkan tingkat \textit{evasion rate} hingga 94\% terhadap EDR modern, karena kemampuannya menyembunyikan \textit{payload} dalam memori dan mengubah \textit{behavioral signature} secara dinamis melalui \textit{obfuscation} \parencite{elghaly2024}. Dalam konteks \textit{dropper}, \textit{fileless execution} menjadi metode \textit{delivery} utama karena memungkinkan \textit{payload spyware} diaktifkan tanpa menulis file ke disk, yang secara langsung relevan dengan prototipe \textit{stealth} yang dikembangkan dalam penelitian ini.

Tantangan utama \textit{fileless malware} terletak pada keterbatasan sistem deteksi yang masih bergantung pada \textit{file artifacts}, di mana \textit{memory forensics} dan \textit{behavioral analysis} menjadi satu-satunya pertahanan efektif namun \textit{resource-intensive}. Laporan industri mencatat bahwa 77\% organisasi mengalami serangan \textit{fileless} pada 2019, dengan kerugian rata-rata mencapai jutaan dolar per insiden karena sulitnya \textit{attribution} dan \textit{remediation}. Penelitian \textit{Sudhakar dan Kumar} \parencite{sudhakar2020} secara kritis menyoroti bahwa dari 20 metode deteksi \textit{fileless} yang dianalisis, hanya 30\% efektif terhadap varian terbaru yang menggabungkan \textit{process injection} dengan \textit{encrypted C2 communication}. \textit{Gap} ini menjadi dasar metodologi penelitian ini, yang mengintegrasikan \textit{fileless techniques} dalam \textit{dropper} prototipe untuk menguji apakah \textit{anti-spyware} mampu mendeteksi eksekusi \textit{in-memory payload delivery} sebelum \textit{spyware} aktif sepenuhnya.

Relevansi \textit{fileless malware} dengan penelitian ini sangat strategis karena \textit{dropper spyware} mode \textit{stealth} yang dikembangkan mengadopsi prinsip yang sama: operasi tanpa \textit{disk footprint} menggunakan Rust untuk \textit{memory management} tingkat rendah. Dengan mengukur \textit{detection rate} pada tahap \textit{payload activation} melalui \textit{fileless execution}, penelitian ini memberikan bukti empiris tentang efektivitas EDR dan \textit{anti-spyware} terhadap ancaman yang mendominasi 40\% serangan \textit{enterprise} pada 2024. Tanpa pemahaman mendalam tentang \textit{fileless malware} sebagai fondasi \textit{dropper} modern, evaluasi keamanan akan gagal mengidentifikasi celah kritis pada tahap eksekusi yang paling rentan terhadap \textit{evasion} \parencite{elghaly2024,sudhakar2020}.

\section{Teknik Evasi Malware}
Teknik evasi \textit{malware} berkembang pesat sebagai respons terhadap kemajuan sistem deteksi keamanan. \textit{Malware}, khususnya \textit{dropper spyware} mode \textit{stealth}, menggunakan berbagai teknik untuk menyembunyikan keberadaannya dan meningkatkan peluang lolos dari deteksi \textit{anti-malware} dan \textit{Endpoint Detection and Response} (EDR). Salah satu teknik utama adalah \textit{packing} dan \textit{encryption}, yang mengubah \textit{signature} file sehingga tidak cocok dengan basis data tanda tangan \textit{malware} konvensional. Selain itu, \textit{obfuscation} kode dan data digunakan untuk mengacak struktur internal \textit{malware}, menyulitkan analisis statis oleh \textit{antivirus} \parencite{chatzoglou2023}.

Lebih lanjut, mekanisme \textit{fileless execution} memungkinkan \textit{malware} beroperasi sepenuhnya di memori, tanpa menulis file fisik ke disk sehingga sulit dideteksi oleh pemindai file tradisional. \textit{Malware} juga mengimplementasikan \textit{anti-analysis mechanisms} seperti deteksi \textit{virtual machine} (VM), \textit{sandbox}, dan \textit{debugger}, di mana \textit{malware} memeriksa lingkungan eksekusi untuk menentukan apakah sedang dianalisis, dan mengubah perilaku atau menunda eksekusi agar terhindar dari analisis dinamis. Teknik ini termasuk \textit{environment checking}, \textit{sleeping}, dan \textit{API hooking} untuk menghindari pemantauan \parencite{elghaly2024,chatzoglou2023}.

Selain itu, \textit{malware} modern menggunakan komunikasi \textit{Command and Control} (C2) yang terenkripsi dan tersembunyi untuk menghindari inspeksi lalu lintas jaringan. Teknik \textit{steganografi} dan \textit{polymorphism} juga diterapkan untuk menyamarkan komunikasi dan \textit{payload} guna mencegah deteksi berbasis pola trafik jaringan. Studi terbaru menunjukkan bahwa perpaduan beberapa teknik evasi secara simultan meningkatkan tingkat keberhasilan serangan sampai 80\%-90\% dalam melewati deteksi produk keamanan komersial \parencite{koutsokostas2021,elghaly2024}.

Dalam konteks penelitian ini, teknik evasi tersebut menjadi fokus utama karena \textit{dropper spyware} mode \textit{stealth} yang dikembangkan menggabungkan banyak teknik evasi secara integratif. Evaluasi kapabilitas \textit{anti-spyware} dilakukan dengan menguji keberhasilan deteksi terhadap \textit{dropper} yang menggunakan \textit{packing}, \textit{fileless payload delivery}, \textit{anti-VM}, \textit{anti-debug}, serta komunikasi \textit{C2} terenkripsi. Studi ini mengisi celah pada evaluasi sistem keamanan yang selama ini hanya menilai teknik evasi secara parsial, dengan pendekatan holistik yang merefleksikan realita ancaman saat ini \parencite{chatzoglou2023,elghaly2024}.

\section{Deteksi Malware dan Teknologi Anti-Malware}
Deteksi \textit{malware} mengalami evolusi dari pendekatan \textit{signature-based} yang mengandalkan pencocokan \textit{hash} dan pola kode tetap, menuju analisis perilaku (\textit{behavioral analysis}) dan \textit{machine learning} untuk mengenali ancaman \textit{zero-day}. Sistem \textit{anti-malware} konvensional efektif terhadap \textit{malware} statis, namun gagal terhadap varian polimorfik dan \textit{fileless} yang mengubah \textit{signature} secara dinamis atau tidak meninggalkan artefak disk. \textit{Dynamic analysis} melalui \textit{sandbox} mencoba mengeksekusi \textit{malware} dalam lingkungan terkontrol, tetapi rentan terhadap \textit{sandbox evasion} seperti \textit{time-based sleeping} dan \textit{VM artifact detection} yang membuat \textit{malware} mengubah perilaku saat dianalisis \parencite{sudhakar2020,chatzoglou2023}.

\textit{Endpoint Detection and Response} (EDR) muncul sebagai solusi generasi berikutnya yang mengintegrasikan \textit{monitoring real-time}, \textit{threat hunting}, dan \textit{automated response}. Berbeda dengan \textit{antivirus} tradisional, EDR fokus pada \textit{behavioral indicators of compromise} (IoC) seperti \textit{unusual process spawning}, \textit{memory injection}, dan \textit{network beaconing} yang menjadi ciri \textit{dropper spyware}. Namun, studi empiris mengungkap keterbatasan signifikan: EDR modern masih menghasilkan \textit{false positive} tinggi (hingga 30\%) dan gagal mendeteksi 50\% \textit{fileless malware} yang menggunakan LOLBins sah seperti \textit{PowerShell} untuk \textit{C2 communication} \parencite{elghaly2024,ainslie2023}.

\textit{Machine learning} meningkatkan akurasi deteksi melalui analisis fitur seperti urutan \textit{API call}, entropi \textit{payload}, dan pola aliran jaringan, namun rentan terhadap \textit{adversarial attacks} di mana \textit{malware} mengoptimalkan \textit{input} untuk mengecoh model ML. Penelitian menunjukkan bahwa model ML berbasis \textit{deep learning} hanya mencapai 85\% \textit{accuracy} terhadap \textit{dropper} dengan \textit{multi-evasion techniques}, sementara \textit{signatureless detection} masih kalah efektif terhadap \textit{obfuscated payloads} \parencite{reddyvari2024,koutsokostas2021}. \textit{Gap} kritis ini terletak pada kurangnya evaluasi komprehensif terhadap \textit{dropper} sebagai \textit{initial access vector}, di mana sebagian besar studi fokus pada \textit{malware} pasca-infeksi daripada pencegahan pada tahap \textit{delivery}.

Dalam konteks penelitian ini, evaluasi terhadap produk \textit{anti-spyware} dan EDR dilakukan dengan prototipe \textit{dropper} yang mensimulasikan serangan realistis menggunakan Rust, mengukur \textit{detection rate} pada setiap tahap: \textit{unpacking}, \textit{environment checking}, \textit{fileless execution}, dan \textit{payload activation}. Temuan dari pengujian ini akan mengungkap sejauh mana teknologi saat ini mampu menangkal \textit{dropper spyware} mode \textit{stealth}, mengidentifikasi \textit{false negative patterns}, dan memberikan rekomendasi untuk pengembangan \textit{behavioral signatures} yang lebih adaptif terhadap ancaman evolusioner \parencite{chatzoglou2023,elghaly2024}. Analisis ini krusial karena keberhasilan \textit{dropper} menentukan efektivitas keseluruhan \textit{spyware}, menjadikan tahap \textit{initial access} sebagai titik lemah paling strategis dalam rantai pertahanan siber.
