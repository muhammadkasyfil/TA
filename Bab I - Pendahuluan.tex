% ==========================================
% BAB I PENDAHULUAN
% ==========================================
\chapter{PENDAHULUAN}

% --- Latar Belakang ---
\section{Latar Belakang}

Perkembangan teknologi informasi yang pesat turut membawa peningkatan ancaman keamanan siber, terutama berupa malware yang semakin canggih dan sulit dideteksi \parencite{sudhakar2020}. Salah satu varian berbahaya adalah \textit{spyware}, yang merupakan perangkat lunak berbahaya dirancang untuk memantau dan mengumpulkan informasi sensitif tanpa izin, sehingga menimbulkan ancaman serius terhadap privasi dan keamanan pengguna \parencite{reddyvari2024}. Dalam rantai infeksi \textit{spyware} modern, \textit{dropper} memainkan peran penting sebagai komponen pengantar dan eksekutor malware. \textit{Dropper} bertugas untuk mendistribusikan, membuka kemasan (\textit{unpack}), serta menjalankan \textit{payload} \textit{spyware} dalam sistem target dengan teknik yang mampu menghindari deteksi oleh \textit{anti-malware} konvensional \parencite{chatzoglou2023}.

Evolusi ini menunjukkan perpindahan dari ancaman tradisional berbasis berkas menuju serangan yang lebih canggih dan tersembunyi seperti \textit{fileless malware} dan \textit{spyware} mode \textit{stealth}. \textit{Dropper} saat ini tidak hanya berfungsi sebagai pengantar biasa, tetapi juga dilengkapi fitur \textit{anti-debugging}, \textit{anti-virtual machine}, serta teknik \textit{stealth} lainnya yang membuatnya sangat tahan terhadap mekanisme deteksi dan pencegahan \parencite{elghaly2024}. \textit{Dropper} juga bisa bersifat \textit{fileless}, beroperasi langsung di memori tanpa meninggalkan jejak file yang bisa dipindai, sehingga menjadi tantangan besar bagi sistem keamanan \parencite{sudhakar2020}. Keberhasilan \textit{dropper} menentukan efektivitas infeksi \textit{spyware} secara keseluruhan, termasuk kemampuan \textit{persistensi} dan pencurian data \parencite{kareem2024}.

Kondisi keamanan siber saat ini dicirikan oleh evolusi ancaman yang signifikan. Sistem pertahanan keamanan menghadapi beberapa masalah kritis. Pertama, \textit{anti-malware} tradisional masih mengandalkan pencocokan tanda tangan (\textit{signature}), yang dapat dihindari dengan teknik \textit{packing}, enkripsi, dan \textit{obfuscation} untuk mengubah \textit{signature} digital \textit{payload}. Kedua, \textit{fileless malware} memanfaatkan komponen sah sistem operasi seperti \textit{PowerShell} atau \textit{WMI} untuk menjalankan kode berbahaya langsung di memori tanpa menulis berkas ke media penyimpanan, sehingga sulit dideteksi oleh \textit{anti-malware} konvensional \parencite{sudhakar2020}. Ketiga, penelitian menunjukkan bahwa metode \textit{evasion} yang relatif sederhana, seperti enkripsi, injeksi proses, dan penambahan \textit{junk data} ke berkas eksekusi, terbukti sangat efektif dalam menghindari deteksi \parencite{chatzoglou2023}. Evolusi ini menciptakan celah kritis dalam kemampuan deteksi sistem \textit{anti-malware} dan \textit{Endpoint Detection and Response} (EDR) yang tersedia di pasar.

Penelitian ini bertujuan mengkaji kapabilitas \textit{anti-spyware} dalam mendeteksi dan menangkal \textit{dropper spyware} baru yang menerapkan teknik evasi lanjutan. Kajian ini penting untuk mengetahui sejauh mana solusi keamanan saat ini mampu menghadapi ancaman yang semakin kompleks, dan memberikan dasar bagi pengembangan metode pertahanan yang lebih adaptif dan tangguh. Dengan mengacu pada studi dan analisis malware dari berbagai literatur terkini, penelitian ini juga akan menguji efektivitas berbagai produk \textit{anti-spyware} komersial terhadap varian \textit{dropper} tersebut dalam lingkungan simulasi yang realistis.

% --- Rumusan Masalah ---
\section{Rumusan Masalah}
Berdasarkan latar belakang di atas, rumusan masalah penelitian difokuskan pada \textit{dropper spyware} mode \textit{stealth} sebagai mekanisme serangan tahap awal (\textit{initial access}). Pertanyaan penelitian yang diajukan adalah:
\begin{enumerate}
  \item Seberapa efektif sistem \textit{anti-malware} dan \textit{Endpoint Detection \& Response} (EDR) dalam mendeteksi aktivitas \textit{dropper spyware} mode \textit{stealth} yang berfungsi sebagai \textit{initial access and delivery mechanism} pada lingkungan uji terkontrol?
  \item Bagaimana pendekatan paling efektif untuk mengembangkan \textit{dropper spyware stealth} yang mampu menghindari deteksi \textit{anti-malware} dengan memanfaatkan teknik evasi modern seperti \textit{fileless execution}, \textit{in-memory payload delivery}, dan pemeriksaan \textit{anti-analysis}?
  \item Bagaimana pengaruh teknik evasi, misalnya \textit{obfuscation}, enkripsi \textit{payload}, \textit{sandbox evasion}, dan deteksi \textit{anti-VM}, terhadap kemampuan deteksi produk \textit{anti-malware} dan EDR komersial saat ini?
  \item Celah keamanan spesifik apa saja yang muncul pada produk \textit{anti-malware} maupun EDR ketika dihadapkan pada \textit{dropper spyware} kustom yang dirancang dengan teknik penghindaran deteksi terintegrasi?
\end{enumerate}

% --- Tujuan ---
\section{Tujuan}
Secara umum penelitian ini bertujuan mengukur efektivitas sejumlah produk \textit{anti-malware} dan EDR komersial dalam mendeteksi \textit{dropper spyware} mode \textit{stealth} yang bertindak sebagai vektor serangan awal. Tujuan khusus yang hendak dicapai yaitu:
\begin{enumerate}
  \item Menganalisis dan mengidentifikasi teknik evasi \textit{dropper spyware} paling efektif untuk menghindari deteksi \textit{anti-malware} dan EDR, mencakup kategori \textit{file-based}, \textit{network-based}, \textit{script-based}, \textit{downloader-based}, serta \textit{non-persistent}.
  \item Mengembangkan prototipe \textit{dropper spyware} mode \textit{stealth} yang mengintegrasikan teknik evasi modern, seperti \textit{fileless payload delivery}, eksekusi di memori, pemeriksaan \textit{anti-VM}/\textit{anti-debugging}, komunikasi \textit{C2} terenkripsi, dan modul \textit{spyware} aktif menggunakan bahasa pemrograman Rust.
  \item Melakukan pengujian komparatif prototipe terhadap beberapa produk \textit{anti-malware} dan EDR terkemuka guna mengamati keberhasilan \textit{payload delivery} serta respons deteksi pada setiap tahap eksekusi.
  \item Mengidentifikasi celah keamanan dan kelemahan deteksi pada tiap produk, kemudian merumuskan rekomendasi teknis maupun strategis untuk meningkatkan kapabilitas mitigasi serangan \textit{initial access} berbasis \textit{dropper}.
\end{enumerate}

Kriteria keberhasilan penelitian meliputi:
\begin{itemize}
  \item Prototipe \textit{dropper} berhasil mengimplementasikan sedikitnya tiga teknik evasi terintegrasi dan terbukti lolos dari deteksi minimal pada satu produk \textit{anti-malware} yang diuji
  \item Hasil eksperimen menunjukkan perbedaan signifikan antar produk dalam mendeteksi \textit{dropper} dan mengungkap tahap eksekusi yang paling sulit diawasi
\end{itemize}

% --- Batasan Masalah ---
\section{Batasan Masalah}
Penelitian ini memiliki batasan yang ditetapkan untuk memastikan fokus dan kelayakan penelitian.

\textbf{Batasan Penelitian}
\begin{enumerate}
  \item Batasan umum: Seluruh eksperimen dilaksanakan di lingkungan simulasi terkontrol dengan skenario yang telah ditetapkan. Ruang lingkup hanya sampai pemastian \textit{payload} berhasil dijalankan pada sistem target tanpa mengevaluasi dampak pasca-infeksi.
  \item Batasan desain pengujian: Penelitian berfokus pada \textit{dropper spyware} mode \textit{stealth} yang memanfaatkan teknik \textit{fileless execution}, \textit{in-memory payload}, serta pemeriksaan \textit{anti-analysis}. Metode distribusi lain seperti \textit{worm} atau \textit{supply-chain attack} tidak dianalisis secara detail.
  \item Batasan implementasi: Pengujian hanya menilai kemampuan deteksi dan respons \textit{anti-malware}/EDR hingga tahap \textit{payload} aktif, tanpa melakukan analisis rinci terhadap \textit{payload} lanjutan atau serangan lanjutan setelah \textit{dropper} selesai bertugas.
  \item Batasan lingkungan: Penelitian tidak melibatkan sistem nyata atau jaringan publik. Semua pengujian dilakukan dalam lingkungan virtual terisolasi untuk mencegah penyalahgunaan dan penyebaran malware.
  \item Batasan produk yang diuji: Evaluasi dilakukan terhadap produk \textit{anti-malware} dan EDR yang tersedia dan dapat diakses secara legal untuk tujuan akademis atau \textit{trial period}.
  \item Tugas akhir ini dikerjakan secara kelompok dengan anggota penelitian sebagai berikut:
  \begin{itemize}
    \item Nathaniel Liady
    \item M. Kasyfil Aziz
    \item Audra Zelvania Putri Harjanto
    \item Khayla Belva Annandira
  \end{itemize}
\end{enumerate}



% --- Metodologi ---
\section{Metodologi}
Metodologi penelitian mengacu pada \textit{Design Research Methodology} (DRM) yang diperkenalkan oleh Blessing dan Chakrabarti \parencite{blessing2009}. Kerangka ini dipilih karena menyajikan proses iteratif yang terstruktur untuk merancang sekaligus mengevaluasi artefak teknis dalam konteks penelitian pendesainan sistem yang kompleks. DRM telah terbukti efektif dalam mengintegrasikan riset akademis dengan pengembangan praktis, sehingga cocok untuk penelitian yang mengkombinasikan analisis keamanan siber dengan pengembangan prototipe malware untuk tujuan defensif.

\begin{figure}[h]
  \centering
  \includegraphics[width=0.6\textwidth]{image/drm.png}
  \caption{Kerangka kerja DRM\label{fig:drm-framework}}
\end{figure}

Kerangka DRM terdiri atas empat fase utama yang saling terkait dan berulang.

\begin{enumerate}
  \item Research Clarification (RC)\\
  Fase pertama adalah Research Clarification (RC), yang mengidentifikasi kesenjangan antara evolusi teknik \textit{dropper} modern dan keterbatasan sistem deteksi yang ada saat ini. Pada tahap ini, dilakukan studi literatur mendalam terhadap laporan industri keamanan siber terkini untuk memahami tren ancaman terbaru, analisis publikasi akademis mengenai teknik evasi malware termasuk \textit{fileless execution} dan \textit{anti-analysis mechanisms}, studi kasus analisis terhadap serangan \textit{dropper} nyata yang telah didokumentasikan, serta wawancara dengan praktisi keamanan siber untuk mengkonfirmasi relevansi masalah. Hasil dari RC adalah rumusan masalah yang terdefinisi jelas, urgensi penelitian yang berdasar pada bukti mutakhir, dan identifikasi parameter kritis yang perlu dievaluasi.
  \item Descriptive Study I (DS-I)\\
  Fase kedua adalah Descriptive Study I (DS-I), yang menelaah kondisi eksisting melalui analisis karakteristik \textit{dropper} modern dalam \textit{cyber kill chain} yang komprehensif. Fokus DS-I ada pada identifikasi mekanisme \textit{fileless payload delivery} dan bagaimana \textit{dropper} memanfaatkan memori sistem untuk eksekusi, analisis teknik \textit{anti-analysis} seperti deteksi VM, deteksi \textit{sandbox}, dan \textit{anti-debugging} untuk memahami bagaimana \textit{dropper} menghindari analisis otomatis, studi terhadap \textit{obfuscation} string dan komunikasi \textit{C2} terenkripsi untuk memahami bagaimana \textit{dropper} menyembunyikan identitasnya, pemetaan metode distribusi \textit{dropper} seperti \textit{phishing}, \textit{drive-by download}, dan dokumen berbahaya, serta rekonstruksi rantai serangan \textit{dropper} dari \textit{initial access} hingga \textit{command-and-control}. Hasil DS-I berupa pernyataan persyaratan fungsional dan non-fungsional yang detail, serta panduan untuk fase desain berikutnya.
  \item Prescriptive Study (PS)\\
  Fase ketiga adalah Prescriptive Study (PS), yang merupakan tahap inti di mana solusi dirancang dan dikembangkan berdasarkan temuan DS-I. Pada tahap ini dilakukan perancangan arsitektur \textit{dropper spyware} mode \textit{stealth} yang mengintegrasikan \textit{multiple evasion techniques} secara kohesif, implementasi prototipe menggunakan bahasa Rust yang menyediakan kontrol \textit{low-level} untuk operasi \textit{fileless} sambil mempertahankan \textit{safety features}, pengembangan modul-modul utama termasuk \textit{payload builder}, manajemen \textit{in-memory execution}, pemeriksaan \textit{anti-VM}/\textit{anti-debug}, kanal komunikasi terenkripsi, dan modul \textit{spyware}, serta melakukan pengujian awal secara iteratif untuk memastikan setiap teknik evasi bekerja sesuai spesifikasi. Selama PS, setiap komponen diuji secara individual dan kemudian diintegrasikan untuk pengujian sistem secara menyeluruh.
  \item Descriptive Study II (DS-II)\\
  Fase keempat dan terakhir adalah Descriptive Study II (DS-II), yang mengevaluasi prototipe yang telah dikembangkan melalui eksperimen yang komprehensif pada beberapa lingkungan uji dan produk \textit{anti-malware}/EDR komersial. Parameter evaluasi pada DS-II mencakup analisis tahap deteksi atau kelolosan (\textit{evasion}) aktivitas \textit{dropper} oleh berbagai \textit{anti-malware}, verifikasi keberhasilan \textit{payload delivery} di memori tanpa meninggalkan artefak disk, pengukuran efektivitas masing-masing teknik evasi dalam menghindari deteksi, analisis performa dan latensi eksekusi, serta identifikasi pola perilaku \textit{dropper} yang terdeteksi versus yang terlewat. Data kuantitatif berupa \textit{detection rate} dan \textit{evasion rate} dibandingkan antar produk \textit{anti-malware} yang diuji. Data kualitatif seperti pola perilaku \textit{dropper} dan jejak forensik dianalisis untuk memahami penyebab lolosnya deteksi. Temuan DS-II menjadi dasar rekomendasi peningkatan mekanisme deteksi dan arah penelitian lanjutan untuk memperkuat pertahanan terhadap \textit{initial access} berbasis \textit{dropper}.
\end{enumerate}




