\chapter{DESAIN KONSEP SOLUSI}

\section{Diagram Konseptual Sistem}
Bab ini menyajikan desain konseptual dari solusi yang diusulkan melalui perbandingan alur sistem sebelum (\textit{before}) dan sesudah (\textit{after}) pengembangan prototipe \textit{dropper spyware} mode \textit{stealth}. Diagram konseptual ini menunjukkan perbedaan fundamental antara kondisi keamanan siber saat ini dan mekanisme eksekusi \textit{dropper} dengan \textit{integrated evasion techniques} yang akan dievaluasi.

\subsection{Sistem Sebelum (Before)}
Sistem deteksi \textit{anti-malware} saat ini beroperasi dengan asumsi bahwa \textit{malware} akan meninggalkan jejak yang terdeteksi melalui \textit{signature matching} atau \textit{behavioral anomalies}. Alur sistem sebelum dapat dijelaskan sebagai berikut. Pertama, ketika user menjalankan file executable, \textit{antivirus} melakukan \textit{on-access scan} dengan membandingkan \textit{hash} file terhadap database \textit{malware} yang diketahui. Jika \textit{hash} cocok, file diblokir dan user diberi notifikasi. Jika \textit{hash} tidak cocok (file \textit{unknown}), sistem melanjutkan dengan \textit{behavioral monitoring} pada tingkat file system dan API calls.

Selama \textit{behavioral monitoring}, sistem memonitor indikator-indikator mencurigakan seperti \textit{file writes} ke \textit{system directories}, \textit{registry modifications}, atau \textit{network connections} yang abnormal. Jika aktivitas mencurigakan terdeteksi, sistem mengeluarkan \textit{alert} kepada user. Namun, jika tidak ada \textit{behavioral anomalies} yang terdeteksi, misalnya karena \textit{dropper} menggunakan \textit{fileless execution}, \textit{in-memory operations}, atau \textit{anti-analysis techniques}, maka \textit{malware} beroperasi tanpa terdeteksi. Diagram alur sistem \textit{before} disajikan pada Gambar~\ref{fig:before}.

\begin{figure}[h]
  \centering
  \includegraphics[width=0.7\textwidth]{image/before.png}
  \caption{Alur Deteksi Anti-Malware Sebelum (Before)}
  \label{fig:before}
\end{figure}

\subsection{Sistem Sesudah (After)}
Sistem sesudah menggambarkan mekanisme eksekusi prototipe \textit{dropper spyware} mode \textit{stealth} dalam lingkungan lab terisolasi dengan \textit{full instrumentation} dan \textit{monitoring}. Alur sistem sesudah dimulai dengan peneliti men-\textit{deploy dropper binary} ke target VM di \textit{lab environment} yang terisolasi. Setelah eksekusi, \textit{dropper} melakukan inisialisasi dan \textit{self-verification} untuk memastikan biner tidak \textit{corrupt}.

Langkah berikutnya adalah \textit{deobfuskasi} dari \textit{obfuscated code sections} yang telah di-\textit{encrypt}, memastikan \textit{core logic} dapat dijalankan \textit{in-memory}. Tahap krusial selanjutnya adalah \textit{environment checking}, di mana \textit{dropper} melakukan \textit{comprehensive checks} untuk mendeteksi kehadiran VM hypervisor, \textit{sandbox tools}, \textit{debugger}, atau \textit{analysis tools}. Jika \textit{environment} terdeteksi sebagai \textit{unsuitable} (VM atau \textit{sandbox} atau \textit{debugger}), \textit{dropper} melakukan \textit{graceful exit} tanpa meninggalkan \textit{artifacts} yang \textit{observable}, sehingga \textit{monitoring systems} tidak menangkap \textit{payload execution} yang meragukan.

Jika \textit{environment} dipastikan aman, \textit{dropper} melanjutkan ke tahap dekripsi \textit{payload} menggunakan AES-256 dengan \textit{key} yang di-\textit{derive} dari karakteristik sistem lokal. Setelah \textit{payload} berhasil didekripsi dan divalidasi, \textit{payload} disimpan dan dieksekusi secara \textit{in-memory} menggunakan teknik \textit{process injection} ke proses \textit{legitimate}. Setelah \textit{payload} berhasil dijalankan, \textit{dropper} melakukan \textit{cleanup} yang \textit{comprehensive} untuk menghapus dirinya dari memori dan tidak meninggalkan \textit{persistence mechanisms}. Diagram alur sistem \textit{after} disajikan pada Gambar~\ref{fig:after}.

\begin{figure}[h]
  \centering
  \includegraphics[width=0.7\textwidth]{image/after.png}
  \caption{Alur Eksekusi Dropper Spyware Mode Stealth (After)}
  \label{fig:after}
\end{figure}

\subsection{Perbandingan Sistem Before dan After}
Perbandingan antara sistem \textit{before} dan \textit{after} menunjukkan perbedaan fundamental dalam pendekatan deteksi versus eksekusi \textit{malware} dengan \textit{integrated evasion techniques}. Tabel~\ref{tab:perbandingan-before-after} merangkum dimensi-dimensi kunci perbandingan.

\begin{longtable}{p{3.0cm}p{5.2cm}p{6.0cm}}
  \caption{Perbandingan Sistem Sebelum dan Sesudah Pengembangan\label{tab:perbandingan-before-after}}\\
  \toprule
  \textbf{Aspek} & \textbf{Sistem Before} & \textbf{Sistem After} \\
  \midrule
  \endfirsthead
  \toprule
  \textbf{Aspek} & \textbf{Sistem Before} & \textbf{Sistem After} \\
  \midrule
  \endhead
  \midrule
  \multicolumn{3}{r}{\textit{bersambung}} \\
  \endfoot
  \bottomrule
  \endlastfoot
  Fokus Deteksi & \textit{Signature-based} pada \textit{hash} file dan \textit{behavioral anomalies} pada disk atau \textit{registry} & \textit{Execution flow} ter-\textit{instrumentasi} dengan \textit{full visibility} pada setiap tahap dari \textit{initialization} hingga \textit{payload execution} \\
  Jenis Malware Target & \textit{Malware} generik dengan minimal \textit{evasion techniques} & \textit{Dropper spyware} mode \textit{stealth} dengan \textit{integrated multiple evasion techniques} (enkripsi, \textit{anti-VM}, \textit{anti-debug}, \textit{fileless execution}) \\
  Environment & \textit{Production systems} dengan \textit{real user activity} & Lab terisolasi dengan \textit{controlled environment} dan \textit{predictable conditions} \\
  Blind Spots & \textit{Fileless execution}, \textit{in-memory operations}, \textit{environment checking}, \textit{behavioral stealth} & Minimal, setiap tahap dipantau melalui Sysmon, \textit{anti-malware logs}, \textit{network capture}, dan \textit{memory analysis} \\
  Visibility Peneliti & Terbatas pada \textit{detection alerts}; tahap eksekusi \textit{malware} tidak teramati & \textit{Comprehensive}, peneliti dapat mengidentifikasi tahap eksekusi spesifik mana yang terdeteksi atau lolos dari setiap produk \textit{anti-malware} atau EDR \\
  Output Evaluasi & \textit{Detection} atau \textit{Non-Detection} binary result & \textit{Quantified metrics} (\textit{Detection Rate}, \textit{Evasion Rate}, \textit{Time-to-Detect}, \textit{Anti-Analysis Efficacy}, \textit{Forensic Artifacts}) dengan \textit{temporal granularity} per tahap eksekusi \\
\end{longtable}

Tabel IV.1 menunjukkan bahwa sistem \textit{after} menyediakan \textit{environment} penelitian yang terstruktur dan \textit{measurable} untuk mengevaluasi efektivitas \textit{evasion techniques} \textit{dropper} serta mengidentifikasi \textit{gap} spesifik dalam kemampuan deteksi sistem \textit{anti-malware} dan EDR yang ada.

\section{Penjelasan Desain Solusi}
Penjelasan desain solusi menjabarkan rancangan teknis dari setiap komponen modul \textit{dropper} dan bagaimana integrasi modul-modul tersebut menciptakan mekanisme eksekusi yang \textit{stealth} dan sulit dideteksi.

\subsection{Desain Inisialisasi dan Verifikasi}
Pada tahap inisialisasi, \textit{dropper} melakukan \textit{internal integrity checking} untuk memastikan bahwa biner \textit{dropper} tidak \textit{corrupt} atau ter-\textit{tamper} selama eksekusi. Verifikasi dilakukan menggunakan \textit{checksum} atau \textit{keyed hashing} terhadap sections kritis dari biner. Jika \textit{integrity check} gagal, \textit{dropper} melakukan \textit{immediate termination} tanpa melakukan operasi lebih lanjut. Jika \textit{integrity check} lolos, \textit{dropper} melanjutkan ke tahap \textit{unpacking} dari \textit{obfuscated code sections} yang telah di-\textit{encrypt}. \textit{Unpacking} ini mengekspos \textit{core logic} dari \textit{dropper} tanpa meninggalkan \textit{disk artifacts}, karena semua operasi terjadi \textit{in-memory}.

\subsection{Desain Anti-Analysis (Environment Checking)}
Modul \textit{environment checker} melakukan \textit{systematic enumeration} dari karakteristik sistem untuk mengidentifikasi kehadiran \textit{analysis environments}. Enumeration mencakup beberapa teknik berikut.
\begin{itemize}
  \item CPUID-based hypervisor detection
  \item Registry scanning
  \item DLL inspection
  \item Anti-debugging checks
\end{itemize}
Jika salah satu \textit{check} positif mendeteksi indikasi \textit{analysis environment}, \textit{dropper} memberikan \textit{signal} kepada tahap berikutnya untuk melakukan \textit{controlled exit} tanpa \textit{executing payload}. \textit{Controlled exit} ini dirancang untuk tidak meninggalkan \textit{error logs} atau \textit{crash dumps} yang dapat menjadi bukti kehadiran \textit{dropper}.

\subsection{Desain In-Memory Execution (Fileless Loading)}
Modul \textit{Loader \& In-Memory Executor} mengandung \textit{logic} untuk dekripsi \textit{payload}, \textit{memory allocation}, dan \textit{payload execution} menggunakan teknik-teknik \textit{minimal-footprint}. Pertama, \textit{payload} yang telah di-\textit{embed} dalam biner \textit{dropper} dalam \textit{state} terenkripsi menggunakan AES-256 dengan \textit{key} yang di-\textit{derive} dari karakteristik sistem lokal. \textit{Key derivation} menggunakan kombinasi dari CPU serial number, HDD serial number, Windows installation ID, dan system uptime untuk menciptakan \textit{key} yang virtually impossible untuk \textit{predict} tanpa \textit{direct access} ke target sistem.

Dekripsi dilakukan menggunakan \textit{standard AES-256-CBC implementation} dengan IV yang juga di-\textit{derive} dari sistem characteristics. Setelah dekripsi, modul melakukan \textit{integrity checking} terhadap \textit{payload} menggunakan \textit{HMAC-SHA256} untuk memastikan \textit{payload} tidak \textit{corrupt} atau ter-\textit{modified}. Jika \textit{integrity check} gagal, modul melakukan \textit{cleanup} dan \textit{termination}.

Setelah dekripsi sukses, \textit{payload} disimpan di memori yang dialokasikan secara dinamis menggunakan \texttt{VirtualAlloc} dengan \textit{protections} yang appropriate untuk sections yang perlu \textit{execution}). \textit{Allocation} ini dilakukan tanpa menciptakan file di disk atau \textit{registry entries} yang dapat menjadi \textit{forensic artifacts}. Setelah \textit{memory allocation}, \textit{payload} dieksekusi menggunakan teknik \textit{process injection} ke proses \textit{legitimate} yang already running di sistem. Teknik \textit{injection} yang dapat digunakan meliputi \texttt{CreateRemoteThread} dengan \textit{shellcode injection}, \textit{APC injection} dengan \textit{queueing} \textit{user-mode APC} ke target thread, atau \textit{reflective DLL loading} di mana DLL di-\textit{load} langsung ke memori tanpa \textit{registry entries}.

\subsection{Desain Komponen Modul Dropper}
Integrasi dari semua modul dirancang untuk menciptakan alur eksekusi yang kohesif dan \textit{seamless}. Tabel~\ref{tab:komponen-dropper} merangkum komponen-komponen modul utama dan fungsinya.

\begin{longtable}{p{0.8cm}p{4.0cm}p{5.8cm}p{3.6cm}}
  \caption{Komponen Modul Dropper dan Fungsi Evasion\label{tab:komponen-dropper}}\\
  \toprule
  \textbf{No} & \textbf{Modul} & \textbf{Fungsi Utama} & \textbf{Teknik Evasion} \\
  \midrule
  \endfirsthead
  \toprule
  \textbf{No} & \textbf{Modul} & \textbf{Fungsi Utama} & \textbf{Teknik Evasion} \\
  \midrule
  \endhead
  \midrule
  \multicolumn{4}{r}{\textit{bersambung}} \\
  \endfoot
  \bottomrule
  \endlastfoot
  1 & Initialization dan Self-Verify & Melakukan \textit{integrity checking} terhadap biner \textit{dropper} sendiri dan memastikan tidak \textit{corrupt} atau ter-\textit{tamper} & \textit{Code integrity verification} \\
  2 & Code Unpacking dan Deobfuscation & Melakukan \textit{unpacking} dari \textit{obfuscated code sections} yang telah di-\textit{encrypt} secara \textit{runtime in-memory} & \textit{Obfuscation bypass}, \textit{Unpacking} \\
  3 & Environment Checker & Memeriksa karakteristik VM, \textit{sandbox}, \textit{debugger}, dan \textit{analysis tools} sebelum eksekusi \textit{payload} & \textit{Anti-VM}, \textit{Anti-Sandbox}, \textit{Anti-Debug} \\
  4 & Payload Decryption dan Validation & Melakukan dekripsi \textit{payload} terenkripsi menggunakan AES-256 dengan \textit{key} yang unik per target & \textit{Encryption}, \textit{Payload obfuscation} \\
  5 & In-Memory Loader & Mengalokasikan memori secara dinamis dan mengeksekusi \textit{payload in-memory} menggunakan \textit{process injection} & \textit{Fileless execution}, \textit{Process injection} \\
  6 & Cleanup dan Self-Termination & Menghapus jejak \textit{dropper} dari memori, menutup \textit{handles}, dan \textit{exit gracefully} tanpa meninggalkan \textit{forensic artifacts} & \textit{Non-persistence}, \textit{Anti-forensics} \\
\end{longtable}

Setiap modul dirancang untuk beroperasi independen namun juga saling mendukung, menciptakan \textit{layered defense approach} yang meningkatkan probabilitas \textit{evasion} terhadap deteksi \textit{multi-layer} dari sistem keamanan modern.

\section{Rancangan Lingkungan Pengujian}
Lingkungan pengujian dirancang untuk mensimulasikan kondisi \textit{execution dropper spyware mode stealth} dalam konteks terisolasi yang memungkinkan peneliti untuk mengobservasi dan merekam setiap tahap eksekusi dengan \textit{full visibility}, sekaligus memastikan keamanan dan \textit{isolation} dari jaringan eksternal. Lingkungan pengujian dibangun di atas \textit{single host machine} (laptop pribadi peneliti) dengan menggunakan virtualisasi untuk menciptakan \textit{multiple isolated guest VMs} yang masing-masing dikonfigurasi dengan produk \textit{anti-malware} atau EDR berbeda.

\textit{Isolasi network} dilakukan menggunakan \textit{virtual networking} dengan konfigurasi \textit{host-only} atau NAT, memastikan bahwa guest VMs tidak memiliki akses ke jaringan internet eksternal. Semua komunikasi C2 (\textit{Command and Control}) dimock di dalam \textit{lab network} menggunakan C2 server \textit{dummy} lokal untuk merepresentasikan komunikasi \textit{malicious} tanpa risiko \textit{escape} ke internet. Monitoring pada \textit{multiple layers} dilakukan untuk mengumpulkan data komprehensif tentang perilaku \textit{dropper} dengan cara sebagai berikut.
\begin{itemize}
  \item \textbf{File system layer} menggunakan Sysmon untuk merekam semua \textit{file operations}.
  \item \textbf{Registry layer} menggunakan WMI atau \textit{custom monitoring scripts} untuk merekam \textit{registry changes}.
  \item \textbf{Process layer} menggunakan \textit{API hooking} atau Sysmon untuk merekam \textit{process creation}, \textit{thread creation}, dan \textit{module loading}.
  \item \textbf{Network layer} menggunakan \textit{packet capture tools} (\textit{tcpdump} atau Wireshark) untuk merekam semua \textit{network communications}.
  \item \textbf{Memory layer} menggunakan \textit{kernel debugger} atau \textit{live memory analysis tools} untuk \textit{examining memory contents} dan \textit{process memory maps}.
\end{itemize}

\section{Spesifikasi Mesin Uji}
Pengujian \textit{dropper} dilakukan menggunakan infrastruktur lokal yang terdiri dari satu \textit{host physical} (laptop pribadi peneliti) dan beberapa guest VMs yang dijalankan di atasnya. Spesifikasi \textit{hardware} dan \textit{software} untuk mesin uji disajikan pada Tabel~\ref{tab:spesifikasi-uji}.

\begin{longtable}{p{3.0cm}p{5.8cm}p{5.8cm}}
  \caption{Spesifikasi Lingkungan Uji\label{tab:spesifikasi-uji}}\\
  \toprule
  \textbf{Komponen} & \textbf{Spesifikasi} & \textbf{Keterangan} \\
  \midrule
  \endfirsthead
  \toprule
  \textbf{Komponen} & \textbf{Spesifikasi} & \textbf{Keterangan} \\
  \midrule
  \endhead
  \midrule
  \multicolumn{3}{r}{\textit{bersambung}} \\
  \endfoot
  \bottomrule
  \endlastfoot
  \multicolumn{3}{l}{\textbf{Host Physical (Laptop Pribadi)}} \\
  Processor & Intel Core i5 atau i7 atau AMD Ryzen 5 atau 7 (minimum 4 \textit{cores}) & Untuk menjalankan \textit{multiple VMs} secara bersamaan dengan performa \textit{adequate} \\
  RAM & 16 GB minimum & Alokasi 4–6 GB per guest VM, tersisa untuk host OS dan \textit{monitoring tools} \\
  Storage & SSD 512 GB atau lebih & Untuk host OS, Rust \textit{toolchain}, VirtualBox atau VMware, guest VM images, dan \textit{logging data} \\
  OS Host & Windows 10 Pro atau 21H2 atau lebih atau Windows 11 & Sebagai host OS untuk \textit{hypervisor} dan \textit{development environment} \\
  Hypervisor & VirtualBox 7.0 atau lebih atau VMware Workstation & Untuk membuat dan menjalankan guest VMs terisolasi \\
  \midrule
  \multicolumn{3}{l}{\textbf{Guest VM 1-3}} \\
  OS & Windows 10 Pro (Build 19042 atau lebih) atau Windows 11 (Build 22621 atau lebih) & \textit{Representative} terhadap \textit{enterprise desktop} yang menjadi target \textit{dropper} \\
  vCPU & 2 \textit{cores} & \textit{Sufficient} untuk menjalankan OS dan \textit{single malware execution} \\
  vRAM & 4–6 GB & \textit{Adequate} untuk Windows 10 atau 11 dan \textit{anti-malware software} \\
  Storage & 40–50 GB \textit{virtual disk} & \textit{Space} untuk OS, \textit{anti-malware}, Sysmon, \textit{logging}, dan \textit{temporary test files} \\
  Network & \textit{Host-only virtual switch} & \textit{Isolated network} tanpa akses ke jaringan eksternal; semua \textit{traffic} terikat di \textit{lab network} \\
  Monitoring Tools & Sysmon, Process Explorer, Registry Editor & Untuk merekam dan mengobservasi aktivitas sistem selama \textit{dropper execution} \\
  \midrule
  \multicolumn{3}{l}{\textbf{Anti-Malware atau EDR Products}} \\
  VM 1 & Windows Defender (\textit{built-in}) & \textit{Baseline anti-malware} dengan \textit{behavioral monitoring} \\
  VM 2 & AVG Free atau Kaspersky Free & \textit{Alternative anti-malware} dengan \textit{heuristics} dan \textit{cloud-based detection} \\
  VM 3 & McAfee Personal Security atau Norton LifeLock Trial & \textit{Commercial-grade anti-malware} dengan \textit{advanced behavioral analysis} \\
  \midrule
  \multicolumn{3}{l}{\textbf{Monitoring dan Logging Infrastructure}} \\
  Sysmon Agent & \textit{Deployed} di semua guest VMs & Merekam \textit{process creation}, \textit{file writes}, \textit{registry modifications}, \textit{network connections} pada \textit{Event ID} level dengan \textit{custom rule sets} \\
  Network Monitoring & \textit{Packet capture tools} (\textit{tcpdump} atau Wireshark) & \textit{Running} pada host OS atau \textit{dedicated monitoring VM} untuk \textit{capturing C2 dummy communication} dan \textit{anomalous traffic} \\
  Log Aggregation & \textit{Text files} atau \textit{simple SQLite database} & Kumpulkan \textit{logs} dari Sysmon, \textit{anti-malware}, dan \textit{monitoring tools} ke \textit{centralized location} di host untuk \textit{post-analysis} \\
  Analysis Tools & x64dbg, WinDbg, \textit{strings}, IDA Free & Untuk \textit{debugging dropper execution}, \textit{memory analysis}, dan \textit{binary inspection} di \textit{post-execution} fase \\
\end{longtable}

Spesifikasi ini dirancang untuk \textit{balanced} antara \textit{resource constraints} dari \textit{single laptop} pribadi dan kebutuhan untuk \textit{adequate monitoring} dan \textit{evaluation} dari \textit{multiple anti-malware products} secara \textit{concurrent}. Pengujian dapat dilakukan secara \textit{sequential} jika \textit{resource} terbatas, atau \textit{parallel} jika host memiliki \textit{resource} yang cukup.
