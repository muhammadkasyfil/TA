% ============================================================================================
% BAB III ANALISIS MASALAH
% ============================================================================================
\chapter{ANALISIS MASALAH}
\label{chap:analisis-masalah}

\section{Analisis Kondisi dan Kesenjangan (Current State and Gap Analysis)}
Kondisi keamanan siber saat ini dicirikan oleh evolusi ancaman yang signifikan, bergerak dari \textit{malware} tradisional berbasis berkas menuju serangan yang lebih canggih dan tersembunyi seperti \textit{fileless malware} dan \textit{spyware} mode \textit{stealth}. Evolusi ini menciptakan celah kritis dalam kemampuan deteksi sistem \textit{anti-malware} dan \textit{Endpoint Detection and Response} (EDR) yang tersedia di pasar. Pemahaman mendalam tentang kesenjangan antara kapabilitas ancaman terkini dan sistem pertahanan yang ada menjadi fondasi penting untuk merancang evaluasi penelitian yang komprehensif dan relevan secara praktis.

\subsection{Keterbatasan Deteksi Konvensional}
Sistem pertahanan keamanan siber menghadapi beberapa masalah kritis yang membuat sistem rentan terhadap \textit{dropper spyware} mode \textit{stealth}:
\begin{enumerate}[label=\alph*)]
  \item \textbf{Keterbatasan Deteksi Berbasis Tanda Tangan (Signature-Based Detection)}\\
  \textit{Anti-malware} tradisional masih mengandalkan pencocokan \textit{signature} untuk mengidentifikasi \textit{malware} yang diketahui. Namun, \textit{spyware} mode \textit{stealth} menggunakan teknik \textit{packing}, enkripsi, dan \textit{obfuscation} untuk mengubah \textit{signature} digital \textit{payload} sehingga efektif lolos dari deteksi berbasis \textit{signature}. Studi empiris menunjukkan bahwa dari 12 produk \textit{antivirus} yang diuji, hampir separuh hanya mampu mendeteksi kurang dari 50\% varian \textit{malware} yang disamarkan dengan teknik \textit{evasion} sederhana. Keterbatasan ini disebabkan oleh fakta bahwa setiap perubahan minor pada \textit{payload}, bahkan hanya penambahan \textit{junk data} atau modifikasi \textit{packing} akan menghasilkan \textit{hash} yang berbeda, membuat \textit{signature} lama menjadi tidak berlaku.
  \item \textbf{Kegagalan Deteksi \textit{Fileless Execution} dan Analisis Kode}\\
  \textit{Fileless malware} memanfaatkan komponen sah sistem operasi seperti \textit{PowerShell} atau WMI untuk menjalankan kode berbahaya langsung di memori tanpa menulis berkas ke media penyimpanan, sehingga sulit dideteksi oleh \textit{anti-malware} konvensional. Kegagalan alat keamanan dalam menganalisis kode yang dikemas menggunakan alat populer seperti \textit{PyInstaller} terjadi karena \textit{antivirus} tidak dapat memahami konten \textit{bytecode} yang disamarkan. Penelitian menunjukkan bahwa EDR modern masih menghasilkan \textit{false negative rate} hingga 50\% terhadap \textit{fileless malware}, karena fokus \textit{monitoring} masih pada \textit{disk artifacts} dan \textit{registry persistence entries}, bukan aktivitas \textit{in-memory} yang sesungguhnya.
  \item \textbf{Efektivitas Teknik \textit{Evasion} Sederhana}\\
  Penelitian menunjukkan bahwa metode \textit{evasion} yang relatif sederhana, seperti enkripsi, \textit{process injection}, dan penambahan \textit{junk data} ke berkas eksekusi, terbukti sangat efektif dalam menghindari deteksi. Studi empiris menunjukkan bahwa perpaduan tiga teknik \textit{evasion} dasar saja sudah mampu meningkatkan \textit{evasion rate} hingga 80\% terhadap produk \textit{antivirus} komersial. Hal ini mengindikasikan bahwa sistem deteksi masih mengandalkan pendekatan terpisah, mempertimbangkan setiap teknik \textit{evasion} secara terpisah, daripada analisis holistik terhadap kombinasi \textit{multiple evasion techniques} dalam satu artefak \textit{malware}.
  \item \textbf{Keterbatasan \textit{Behavioral Analysis} dan \textit{Heuristics}}\\
  Meskipun \textit{behavioral analysis} dan \textit{heuristics} telah dikembangkan untuk mengatasi keterbatasan \textit{signature-based detection}, sistem ini masih rentan terhadap \textit{false positive} dan \textit{false negative} yang signifikan. \textit{Anti-VM detection}, \textit{anti-debugging}, dan \textit{sandbox evasion} memungkinkan \textit{dropper} untuk menyamarkan perilaku mencurigakan atau mengubah eksekusi saat dideteksi adanya analisis otomatis. Penelitian terbaru menemukan bahwa \textit{dropper} dengan \textit{anti-analysis mechanisms} terintegrasi mampu melewati 94\% \textit{sandbox} publik dan 80\% EDR \textit{enterprise-grade} dalam pengujian awal.
\end{enumerate}

\subsection{Analisis Kesenjangan Desain \textit{Dropper} (Gap Analysis)}
Berdasarkan keterbatasan deteksi yang telah diidentifikasi, analisis kesenjangan berikut menyoroti bahwa artefak penelitian yang ada belum secara memadai mencakup \textit{dropper} yang mengintegrasikan teknik-teknik \textit{evasive} modern dan tidak konvensional secara simultan. Tabel \ref{tab:gap-analysis} merangkum hubungan antara celah deteksi saat ini dan arah solusi yang diusulkan dalam penelitian ini.

\renewcommand{\arraystretch}{1.15}
\begin{longtable}{p{2.7cm}p{3.2cm}p{3.2cm}p{4.1cm}}
  \caption{Ringkasan kesenjangan desain \textit{dropper} dan arah solusi\label{tab:gap-analysis}}\\
  \toprule
  \textbf{Kesenjangan (Gap)} & \textbf{Kondisi Saat Ini} & \textbf{Celah Deteksi} & \textbf{Arah Solusi Penelitian Ini} \\
  \midrule
  \endfirsthead
  \toprule
  \textbf{Kesenjangan (Gap)} & \textbf{Kondisi Saat Ini} & \textbf{Celah Deteksi} & \textbf{Arah Solusi Penelitian Ini} \\
  \midrule
  \endhead
  \midrule
  \multicolumn{4}{r}{\textit{bersambung}} \\
  \endfoot
  \bottomrule
  \endlastfoot
  Bahasa pemrograman & Penelitian terbatas pada C/C++ atau \textit{Python}/\textit{PowerShell} yang \textit{signature}-nya sudah familiar bagi mesin deteksi & \textit{Signature resistance} yang rendah, mudah diidentifikasi oleh \textit{static analysis} & Menggunakan Rust untuk \textit{dropper} utama; biner Rust memiliki resistensi inheren terhadap \textit{reverse engineering} dan melewati \textit{signature} yang kurang \textit{tuned} \\
  Operasi \textit{in-memory} dan non-\textit{persistence} & Deteksi EDR sering berfokus pada artefak disk serta \textit{persistence} seperti \textit{registry} atau \textit{startup} entry & Fokus deteksi pada \textit{disk artifacts} membuat \textit{fileless malware} lolos; \textit{non-persistent dropper} tidak meninggalkan jejak \textit{persistence} & Menerapkan eksekusi \textit{fileless} penuh \textit{in-memory payload delivery}, menghilangkan artefak disk dan melakukan \textit{self-terminate} untuk meminimalkan \textit{forensic footprint} \\
  Anti-analisis terintegrasi & \textit{Sandbox} publik kerap gagal mendeteksi \textit{malware} yang menunda eksekusi atau memeriksa artefak sistem & \textit{Anti-VM}, \textit{anti-debug}, \textit{environment checking} belum dievaluasi secara terintegrasi dalam satu \textit{dropper} & Mengintegrasikan \textit{anti-VM} (pemeriksaan \textit{driver} atau \textit{registry}), \textit{anti-debugging} (pemeriksaan \textit{PEB flag}), dan \textit{sandbox evasion} \\
  Integrasi \textit{multiple evasion techniques} & Penelitian sebelumnya mengevaluasi teknik \textit{evasion} secara individual atau pasangan dua & Tidak ada evaluasi holistik terhadap \textit{dropper} dengan 5+ teknik \textit{evasion} terintegrasi & Mengembangkan \textit{dropper} dengan minimal 5 teknik \textit{evasion} terintegrasi: enkripsi, \textit{packing}, \textit{anti-VM}, \textit{anti-debug}, \textit{fileless execution} \\
  Evaluasi komparatif sistematis & Evaluasi \textit{anti-malware}/EDR belum spesifik terhadap \textit{dropper} sebagai \textit{initial access vector} & Fokus evaluasi pada \textit{malware} pasca-infeksi, bukan pada fase \textit{critical delivery} & Melakukan pengujian komparatif terhadap 6+ produk \textit{anti-spyware}/EDR untuk mengidentifikasi perbedaan \textit{detection rate} per tahap eksekusi \textit{dropper} \\
\end{longtable}

Kesenjangan-kesenjangan yang telah diidentifikasi menghasilkan beberapa implikasi kritis bagi keamanan siber:
\begin{itemize}
  \item \textbf{Fokus deteksi yang parsial:} sistem \textit{anti-malware} saat ini umumnya dioptimalkan untuk mendeteksi salah satu atau dua teknik \textit{evasion}, namun tidak robust terhadap kombinasi \textit{multiple techniques} yang terintegrasi. Penelitian ini mengisi \textit{gap} tersebut dengan mengembangkan \textit{dropper} yang merepresentasikan ancaman realistis di lapangan.
  \item \textbf{Fase \textit{delivery} yang diabaikan:} mayoritas penelitian fokus pada \textit{malware} pasca-infeksi (\textit{collection}, \textit{exfiltration}) tetapi mengabaikan fase kritis \textit{delivery}. \textit{Dropper} sebagai \textit{initial access vector} menjadi sasaran evaluasi yang kurang layak, padahal keberhasilan fase ini menentukan viabilitas seluruh serangan.
  \item \textbf{Validitas pengujian yang tertanya:} pengujian \textit{anti-malware} menggunakan sampel \textit{malware} yang \textit{trivial} atau tidak realistis dapat memberikan hasil yang tidak akurat. Prototipe \textit{dropper} berbasis Rust dengan \textit{integrated evasion techniques} akan memberikan validitas pengujian yang lebih tinggi.
\end{itemize}

% \section{Metodologi Penelitian (Design Research Methodology DRM)}
% Penelitian ini mengadopsi \textit{Design Research Methodology} (DRM) yang diperkenalkan oleh Blessing dan Chakrabarti sebagai kerangka kerja sistematis untuk merancang, mengembangkan, dan mengevaluasi artefak teknis dalam konteks penelitian desain yang kompleks. DRM dipilih karena menyajikan proses iteratif yang terstruktur untuk mengintegrasikan dasar teoretis yang kuat dengan relevansi praktis, memastikan bahwa prototipe \textit{dropper} yang dikembangkan tidak hanya memenuhi kebutuhan teknis tetapi juga menghasilkan kontribusi akademis yang signifikan.
% \begin{enumerate}
%   \item \textbf{Research Clarification (RC)}\\
%   Fase pertama adalah RC, yang berfokus pada identifikasi masalah utama dan penetapan tujuan penelitian yang jelas. Dalam konteks penelitian ini, RC mengidentifikasi kesenjangan signifikan antara teknik serangan \textit{dropper spyware} modern yang mengintegrasikan \textit{multiple evasion techniques} dan keterbatasan kemampuan deteksi solusi keamanan yang ada. Aktivitas pada fase RC mencakup studi literatur terhadap laporan industri keamanan siber terkini, publikasi akademis mengenai teknik evasi \textit{malware} terbaru, analisis studi kasus serangan \textit{dropper} nyata, serta konsultasi dengan praktisi keamanan siber untuk memvalidasi relevansi masalah. Hasil dari RC adalah rumusan masalah yang terdefinisi jelas, urgensi penelitian yang berdasarkan bukti mutakhir, identifikasi parameter kritis yang perlu dievaluasi, dan \textit{Initial Reference Model} yang menggambarkan kondisi eksisting sistem deteksi terhadap \textit{dropper stealth}.
%   \item \textbf{Descriptive Study I (DS-I)}\\
%   Fase kedua adalah DS-I, yang melakukan studi mendalam terhadap kondisi eksisting melalui analisis karakteristik \textit{dropper} modern dalam \textit{cyber kill chain}. Fokus utama DS-I ada pada analisis \textit{fileless payload delivery mechanisms} dan bagaimana \textit{dropper} memanfaatkan memori sistem untuk eksekusi tanpa \textit{disk footprint}, identifikasi pemeriksaan \textit{anti-analysis} seperti \textit{anti-VM detection}, \textit{anti-debugging}, dan \textit{sandbox evasion} serta mekanisme kerjanya, studi \textit{obfuscation} string dan komunikasi \textit{C2} terenkripsi, pemetaan metode distribusi \textit{dropper} seperti \textit{phishing} dan \textit{drive-by download}, serta rekonstruksi rantai serangan \textit{dropper} dari \textit{initial access} hingga \textit{command-and-control} dengan identifikasi tahap-tahap yang paling sulit dideteksi. DS-I menghasilkan \textit{Descriptive Model} yang komprehensif tentang teknik-teknik \textit{evasion} yang digunakan \textit{dropper} modern, persyaratan fungsional dan non-fungsional yang detail untuk fase \textit{design} berikutnya, serta pemahaman mendalam tentang celah-celah spesifik dalam kemampuan deteksi \textit{anti-malware} dan EDR terhadap setiap teknik \textit{evasion}.
%   \item \textbf{Prescriptive Study (PS)}\\
%   Fase ketiga adalah PS, merupakan tahap inti di mana solusi dirancang dan dikembangkan berdasarkan temuan DS-I. Pada tahap ini dilakukan perancangan arsitektur \textit{dropper spyware} mode \textit{stealth} yang mengintegrasikan \textit{multiple evasion techniques} secara kohesif, implementasi prototipe menggunakan bahasa Rust yang menyediakan kontrol \textit{low-level} untuk operasi \textit{fileless} sambil mempertahankan \textit{safety features} dan resistensi terhadap \textit{reverse engineering}, pengembangan modul-modul utama termasuk \textit{payload builder}, manajemen \textit{in-memory execution}, pemeriksaan \textit{anti-VM}/\textit{anti-debug}, kanal komunikasi \textit{C2} terenkripsi, dan modul \textit{spyware} aktif. Selama PS, setiap komponen diuji secara individual melalui \textit{unit testing} dan kemudian diintegrasikan untuk pengujian sistem secara menyeluruh melalui \textit{integration testing} dalam lingkungan \textit{sandbox} terkontrol sebelum memasuki fase evaluasi formal.
%   \item \textbf{Descriptive Study II (DS-II)}\\
%   Fase keempat dan terakhir adalah DS-II, merupakan fase evaluasi final yang menguji prototipe \textit{dropper} terhadap beberapa lingkungan uji dan produk \textit{anti-malware}/EDR komersial. Parameter evaluasi pada DS-II mencakup analisis tahap deteksi atau kelolosan aktivitas \textit{dropper} oleh berbagai \textit{anti-malware} pada setiap fase eksekusi, verifikasi keberhasilan \textit{payload delivery} di memori tanpa meninggalkan artefak disk, pengukuran efektivitas masing-masing teknik \textit{evasion} dalam menghindari deteksi, analisis performa dan latensi eksekusi, serta identifikasi pola perilaku \textit{dropper} yang terdeteksi versus yang terlewat. Data kuantitatif berupa \textit{detection rate} dan \textit{evasion rate} dibandingkan antar produk \textit{anti-malware} yang diuji, sementara data kualitatif seperti pola perilaku \textit{dropper} dan jejak forensik dianalisis untuk memahami penyebab lolosnya deteksi. Temuan DS-II menjadi dasar rekomendasi peningkatan mekanisme deteksi dan arah penelitian lanjutan.
% \end{enumerate}

\section{Persyaratan Fungsional dan Non-Fungsional}
Persyaratan fungsional dan non-fungsional diturunkan dari analisis kesenjangan yang telah dilakukan pada Bagian \ref{chap:analisis-masalah} dan berfungsi sebagai kriteria desain sekaligus evaluasi untuk memastikan prototipe \textit{dropper} memenuhi standar yang telah ditetapkan. Persyaratan-persyaratan ini mencerminkan kebutuhan teknis untuk menghadapi ancaman \textit{dropper spyware stealth} modern, sekaligus memastikan bahwa evaluasi yang dilakukan terhadap sistem \textit{anti-malware} dan EDR menggunakan \textit{baseline} yang realistis dan representatif.

\renewcommand{\arraystretch}{1.2}
\begin{longtable}{p{0.8cm}p{4.5cm}p{1.8cm}p{6.4cm}}
  \caption{Persyaratan fungsional (Functional Requirements)\label{tab:functional-req}}\\
  \toprule
  \textbf{ID} & \textbf{Persyaratan Fungsional} & \textbf{Keterkaitan CTQ} & \textbf{Deskripsi} \\
  \midrule
  \endfirsthead
  \toprule
  \textbf{ID} & \textbf{Persyaratan Fungsional} & \textbf{Keterkaitan CTQ} & \textbf{Deskripsi} \\
  \midrule
  \endhead
  \midrule
  \multicolumn{4}{r}{\textit{bersambung}} \\
  \endfoot
  \bottomrule
  \endlastfoot
  F1 & Eksekusi \textit{payload} \textit{in-memory} (\textit{fileless}) & CTQ-03 & \textit{Dropper} harus memuat dan mengeksekusi \textit{payload} berbahaya langsung di memori tanpa menulis berkas ke media penyimpanan, memastikan tidak ada \textit{disk artifacts} yang dapat dipindai \\
  F2 & \textit{Self-removal} dan non-\textit{persistence} & CTQ-01, CTQ-03 & Setelah mengeksekusi \textit{payload}, \textit{dropper} harus \textit{self-terminate} dan tidak meninggalkan entri \textit{persistence} seperti \textit{registry keys} atau \textit{startup} folder, meminimalkan \textit{forensic footprint} \\
  F3 & Enkripsi dan dekripsi \textit{payload} & CTQ-01, CTQ-02 & \textit{Dropper} membawa \textit{payload} dalam bentuk terenkripsi di dalam biner utama dan mendekripsinya saat \textit{runtime}, mencegah \textit{reverse engineering} dan deteksi \textit{signature} berbasis \textit{payload} \\
  F4 & Pemeriksaan \textit{anti-VM}/\textit{sandbox} & CTQ-03 & \textit{Dropper} mampu mendeteksi lingkungan \textit{virtual machine} atau \textit{sandbox} sebelum mengeksekusi \textit{payload} melalui pemeriksaan \textit{driver} hypervisor, \textit{registry artifacts}, atau \textit{process names} khas virtualisasi \\
  F5 & Pemeriksaan \textit{anti-debugging} & CTQ-03 & \textit{Dropper} mendeteksi upaya \textit{debugging} melalui pemeriksaan \textit{PEB flag}, penggunaan \textit{debugger tools}, atau \textit{breakpoints}, kemudian mengambil tindakan \textit{evasive} seperti keluar paksa atau mengubah alur eksekusi \\
  F6 & Sinkronisasi eksekusi \textit{payload} & CTQ-03 & Menjamin \textit{payload} dieksekusi sukses melalui teknik-teknik seperti \textit{process injection}, \textit{remote thread creation}, atau \textit{reflective DLL loading}, dengan verifikasi bahwa \textit{payload} berjalan dengan benar di konteks \textit{target process} \\
\end{longtable}

Keenam persyaratan fungsional ini dirancang untuk memastikan bahwa \textit{dropper} prototipe mengintegrasikan teknik-teknik \textit{evasion} yang paling efektif dan sulit dideteksi. F1 dan F2 bersama-sama menciptakan profil \textit{dropper} yang \textit{stealthy} dengan tidak meninggalkan jejak permanen di sistem target. F3 mengatasi keterbatasan \textit{signature-based detection} dengan mengenkripsi \textit{payload}, sehingga setiap \textit{hash} akan berbeda meskipun \textit{payload} sebenarnya sama. F4 dan F5 memastikan \textit{dropper} hanya mengeksekusi dalam lingkungan yang sebenarnya (bukan \textit{sandbox} analisis), sedangkan F6 memvalidasi bahwa mekanisme \textit{delivery payload} benar-benar efektif.

\begin{longtable}{p{0.8cm}p{4.5cm}p{2.3cm}p{6.4cm}}
  \caption{Persyaratan non-fungsional (Non-Functional Requirements)\label{tab:nonfunctional-req}}\\
  \toprule
  \textbf{ID} & \textbf{Persyaratan Non-Fungsional} & \textbf{Aspek} & \textbf{Deskripsi} \\
  \midrule
  \endfirsthead
  \toprule
  \textbf{ID} & \textbf{Persyaratan Non-Fungsional} & \textbf{Aspek} & \textbf{Deskripsi} \\
  \midrule
  \endhead
  \midrule
  \multicolumn{4}{r}{\textit{bersambung}} \\
  \endfoot
  \bottomrule
  \endlastfoot
  N1 & \textit{Stealth} dan \textit{evasiveness} & Keamanan & \textit{Dropper} harus memiliki resistensi tinggi terhadap analisis statis melalui penggunaan Rust dan enkripsi, serta deteksi perilaku melalui operasi \textit{fileless} dan \textit{in-memory execution}, dengan target minimum \textit{evasion rate} 70\% terhadap \textit{anti-malware} komersial \\
  N2 & Efisiensi kinerja & Kinerja & \textit{Dropper} harus beroperasi cepat untuk menghindari deteksi anomali \textit{runtime}, dengan \textit{latency payload execution} kurang dari 5 detik \\
  N3 & Keandalan eksekusi & Keandalan & \textit{Dropper} harus stabil ketika menjalankan operasi memori kritis, memanfaatkan keamanan memori Rust untuk mencegah \textit{crash} atau \textit{memory corruption} yang dapat mengungkap aktivitas berbahaya \\
  N4 & Kompatibilitas platform & Kompatibilitas & \textit{Dropper} harus andal di sistem operasi Microsoft Windows versi 10 dan 11, memastikan evaluasi pada platform target yang relevan dengan mayoritas pengguna \textit{enterprise} \\
\end{longtable}

Persyaratan non-fungsional menentukan karakteristik kualitas \textit{dropper} yang membuatnya \textit{viable} dalam skenario serangan \textit{real-world}. N1 memastikan bahwa \textit{dropper} memiliki resistensi terhadap deteksi, N2 menekankan kinerja agar tidak memicu analisis perilaku, N3 mengurangi risiko \textit{crash} yang meninggalkan \textit{error logs}, dan N4 memastikan relevansi platform.

Berdasarkan persyaratan fungsional dan non-fungsional yang telah diidentifikasi, tiga \textit{Critical To Quality} (CTQ) attributes didefinisikan sebagai berikut:
\begin{itemize}
  \item \textbf{CTQ-01: Payload Concealment (Penyembunyian Payload)}\\
  Mengukur sejauh mana \textit{dropper} mampu menyembunyikan dan melindungi \textit{payload} dari deteksi statis dan dinamis. Metrik: persentase \textit{anti-malware} yang gagal mendeteksi \textit{payload} terenkripsi pada fase \textit{static analysis}; target minimum 70\%.
  \item \textbf{CTQ-02: Signature Evasion (Penghindaran Signature)}\\
  Mengukur efektivitas \textit{dropper} dalam menghindari deteksi berbasis \textit{signature} oleh \textit{multi-engine scanners} seperti \textit{VirusTotal}. Metrik: \textit{detection rate} pada \textit{VirusTotal}, target maksimal 20\% (80\% gagal mendeteksi).
  \item \textbf{CTQ-03: Environment Evasion dan Behavioral Stealth}\\
  Mengukur kemampuan \textit{dropper} untuk menghindari deteksi di lingkungan \textit{sandbox}/VM serta beroperasi secara \textit{stealthy} tanpa memicu anomali perilaku. Metrik: persentase \textit{behavioral analysis tools} yang mendeteksi aktivitas mencurigakan; target maksimal 40\%.
\end{itemize}

\section{Desain Keputusan dan Solusi yang Diusulkan}
\subsection{Alternatif Solusi}
Alternatif solusi untuk mengembangkan \textit{dropper spyware} mode \textit{stealth} dengan kemampuan \textit{evasion} tinggi mencakup beberapa pilihan bahasa pemrograman dan pendekatan teknis. Setiap alternatif memiliki \textit{trade-off} antara kontrol \textit{low-level}, resistensi terhadap \textit{reverse engineering}, performa, dan kemudahan implementasi. Tiga alternatif utama yang dipertimbangkan adalah: Menggunakan C/C++ yang memberikan kontrol penuh terhadap memori dan API Windows tetapi rentan terhadap \textit{memory corruption} dan lebih mudah di-\textit{reverse engineer}. Lalu, menggunakan Python dengan \textit{PyInstaller} yang relatif mudah dikembangkan namun \textit{signature}-nya sudah familiar bagi \textit{antivirus} dan kurang \textit{resistant} terhadap \textit{static analysis}. Terakhir, menggunakan Rust yang menawarkan kombinasi unik antara kontrol \textit{low-level}, \textit{memory safety}, performa tinggi, dan resistensi inheren terhadap \textit{reverse engineering}.

Analisis alternatif solusi tidak hanya melibatkan pertimbangan teknis tetapi juga evaluasi terhadap \textit{capability} setiap bahasa dalam mengimplementasikan persyaratan fungsional dan non-fungsional yang telah ditetapkan. Untuk setiap alternatif, dilakukan penilaian terhadap enam kriteria desain: dukungan enkripsi dan \textit{packing}, kontrol \textit{low-level} untuk \textit{fileless execution}, resistensi terhadap \textit{analisis statis}, kemudahan implementasi teknik \textit{anti-analysis}, performansi eksekusi, dan \textit{ecosystem tools/libraries} yang tersedia. Penilaian ini dilakukan secara sistematis menggunakan metodologi \textit{Multi-Criteria Decision Analysis} (MCDA) dengan \textit{Entropy Weight Method} (EWM) untuk menentukan bobot kriteria secara objektif.

\subsection{Analisis Penelitian Solusi}
Analisis penelitian solusi menitikberatkan pada evaluasi efektivitas pendekatan yang telah terbukti dalam literatur untuk mendeteksi \textit{dropper} dan \textit{malware} \textit{evasive}. Studi menunjukkan bahwa kombinasi deteksi berbasis perilaku (\textit{behavioral analysis}), analisis anomali (\textit{anomaly detection}), dan algoritma \textit{machine learning} meningkatkan tingkat deteksi terhadap \textit{dropper} dengan teknik \textit{stealth} dan \textit{fileless}. Namun, penelitian juga mengungkap bahwa metode-metode ini masih memiliki keterbatasan signifikan ketika dihadapkan pada \textit{dropper} yang mengintegrasikan \textit{multiple evasion techniques} secara simultan.

Implementasi \textit{machine learning} dengan kurasi fitur yang relevan seperti \textit{API call sequences}, entropi \textit{payload}, dan \textit{network flow patterns} terbukti mendongkrak presisi dan \textit{recall} dalam klasifikasi \textit{spyware}, menekankan pentingnya kualitas data dan pemilihan fitur untuk mengurangi \textit{false positive}. Penggunaan \textit{dynamic analysis} melalui \textit{sandbox} untuk mengamati proses \textit{unpacking dropper} yang tersembunyi di memori telah terbukti efektif, namun \textit{sandbox} publik masih rentan terhadap \textit{evasion techniques} seperti \textit{time-based sleeping} dan \textit{VM detection}. Integrasi data \textit{threat intelligence} sekaligus deteksi berbasis perilaku adaptif diperlukan agar sistem mampu mengenali pola serangan baru yang belum pernah dijumpai sebelumnya.

Berdasarkan analisis ini, penelitian ini dirancang untuk mengembangkan \textit{dropper} prototipe yang merepresentasikan ancaman tingkat \textit{advanced} dengan \textit{integrated evasion techniques}, sehingga evaluasi terhadap sistem deteksi yang ada akan menghasilkan temuan yang realistis dan \textit{actionable}. Prototipe akan diuji terhadap kombinasi pendekatan deteksi modern-\textit{signature-based}, \textit{behavior-based}, dan \textit{machine learning}-untuk memberikan gambaran komprehensif tentang efektivitas setiap metode.

\subsection{Metodologi Penilaian Solusi (Multi-Criteria Decision Analysis)}
Untuk menentukan bahasa implementasi optimal bagi \textit{dropper} mode \textit{stealth} yang akan dikembangkan, dilakukan evaluasi \textit{multi-kriteria} terhadap tiga alternatif solusi: Rust, C/C++, dan Python dengan \textit{PyInstaller}. Penilaian ini menggunakan \textit{Multi-Criteria Decision Analysis} (MCDA) dengan \textit{Entropy Weight Method} (EWM), yang mampu menentukan bobot kriteria secara objektif berdasarkan tingkat diversifikasi informasi dalam matriks keputusan. EWM dipilih karena objektivitasnya dalam penentuan bobot tanpa melibatkan subjektivitas \textit{expert judgment}.

\paragraph{Langkah 1: Normalisasi Matriks Keputusan}

Skor mentah dari matriks keputusan ($\mathbf{X}$) dinormalisasi menjadi matriks $\mathbf{P}$ untuk menghilangkan pengaruh unit dan skala pengukuran menggunakan:
\begin{equation}
  P_{ij} = \frac{x_{ij}}{\sum_{i=1}^{m} x_{ij}}
  \label{eq:normalisasi}
\end{equation}
Keterangan: $P_{ij}$ adalah nilai normalisasi kriteria $j$ untuk alternatif $i$, $x_{ij}$ adalah skor mentah, $m$ adalah jumlah alternatif ($m=3$: Rust, C/C++, Python), dan $n$ adalah jumlah kriteria ($n=6$).

\paragraph{Langkah 2: Perhitungan Nilai Entropi}

Nilai entropi ($e_{j}$) dihitung untuk setiap kriteria $j$ guna mengukur tingkat ketidakpastian atau dispersi data:
\begin{equation}
  e_{j} = -k \sum_{i=1}^{m} P_{ij} \ln (P_{ij}), \qquad k = \frac{1}{\ln(m)}
  \label{eq:entropi}
\end{equation}

\paragraph{Langkah 3: Penentuan Bobot Kriteria}

Derajat diversifikasi ($d_{j}$) diperoleh melalui:
\begin{equation}
  d_{j} = 1 - e_{j}
  \label{eq:diversifikasi}
\end{equation}
Bobot relatif ($w_{j}$) untuk setiap kriteria kemudian dihitung dengan:
\begin{equation}
  w_{j} = \frac{d_{j}}{\sum_{j=1}^{n} d_{j}}
  \label{eq:bobot}
\end{equation}
Bobot mencerminkan kontribusi setiap kriteria terhadap pengambilan keputusan, di mana kriteria dengan variabilitas tinggi mendapat bobot lebih besar.

\paragraph{Langkah 4: Perhitungan Skor Komposit Akhir}

Skor komposit akhir ($S_{i}$) untuk setiap alternatif solusi dihitung menggunakan:
\begin{equation}
  S_{i} = \sum_{j=1}^{n} w_{j} P_{ij}
  \label{eq:skor}
\end{equation}
Alternatif dengan skor tertinggi dipilih sebagai solusi yang paling optimal berdasarkan pertimbangan \textit{multi-kriteria} yang objektif.

\subsection{Hasil Analisis Penentuan Solusi}
Perhitungan MCDA-EWM dilakukan terhadap matriks keputusan dengan enam kriteria desain dan tiga alternatif solusi. Kriteria yang dievaluasi adalah: KD-1 (Dukungan Enkripsi/\textit{Packing}), KD-2 (Kontrol \textit{Low-Level Fileless}), KD-3 (Resistensi Analisis Statis), KD-4 (Kemudahan Implementasi \textit{Anti-Analysis}), KD-5 (Performa Eksekusi), dan KD-6 (\textit{Ecosystem Tools/Libraries}). Hasil perhitungan entropi menunjukkan bahwa KD-2 memiliki entropi terendah dan derajat diversifikasi tertinggi, mengindikasikan bahwa kriteria ini paling efektif dalam membedakan ketiga alternatif. Tabel \ref{tab:bobot-ewm} merangkum hasil perhitungan bobot kriteria.

\renewcommand{\arraystretch}{1.15}
\begin{longtable}{p{1.4cm}p{6cm}ccc}
  \caption{Bobot kriteria hasil metode EWM\label{tab:bobot-ewm}}\\
  \toprule
  \textbf{Kode Kriteria} & \textbf{Deskripsi Kriteria} & \textbf{$e_{j}$} & \textbf{$d_{j}$} & \textbf{$w_{j}$} \\
  \midrule
  \endfirsthead
  \toprule
  \textbf{Kode Kriteria} & \textbf{Deskripsi Kriteria} & \textbf{$e_{j}$} & \textbf{$d_{j}$} & \textbf{$w_{j}$} \\
  \midrule
  \endhead
  \midrule
  \multicolumn{5}{r}{\textit{bersambung}} \\
  \endfoot
  \bottomrule
  \endlastfoot
  KD-1 & Dukungan Enkripsi/\textit{Packing} & 0.993 & 0.007 & 0.057 \\
  KD-2 & Kontrol \textit{Low-Level Fileless} & 0.923 & 0.077 & 0.631 \\
  KD-3 & Resistensi Analisis Statis & 0.957 & 0.043 & 0.352 \\
  KD-4 & Kemudahan Implementasi \textit{Anti-Analysis} & 0.989 & 0.011 & 0.090 \\
  KD-5 & Performa Eksekusi & 0.994 & 0.006 & 0.049 \\
  KD-6 & \textit{Ecosystem Tools/Libraries} & 0.998 & 0.002 & 0.016 \\
  \midrule
  \multicolumn{2}{r}{\textbf{Total}} &  & 0.146 & 1.000 \\
\end{longtable}

Nilai bobot menunjukkan bahwa KD-2 memberikan kontribusi terbesar (63.1\%) terhadap pengambilan keputusan, diikuti oleh KD-3 (35.2\%), sedangkan kriteria lainnya memiliki kontribusi kecil.

Skor komposit akhir untuk setiap alternatif disajikan pada Tabel \ref{tab:skor-akhir}. Rust memperoleh skor tertinggi (0.370), diikuti oleh C/C++ (0.334), dan Python (0.296). Rust unggul terutama pada KD-2 dan KD-3 karena kombinasi uniknya antara kontrol \textit{low-level} dan resistensi terhadap \textit{analisis statis}.

\begin{longtable}{p{3cm}c p{8cm}}
  \caption{Skor komposit akhir alternatif solusi\label{tab:skor-akhir}}\\
  \toprule
  \textbf{Alternatif Solusi} & \textbf{Skor Komposit ($S_{i}$)} & \textbf{Keterangan} \\
  \midrule
  \endfirsthead
  \toprule
  \textbf{Alternatif Solusi} & \textbf{Skor Komposit ($S_{i}$)} & \textbf{Keterangan} \\
  \midrule
  \endhead
  \midrule
  \multicolumn{3}{r}{\textit{bersambung}} \\
  \endfoot
  \bottomrule
  \endlastfoot
  Rust & 0.370 & Terpilih - unggul pada KD-2 (kontrol \textit{low-level}) dan KD-3 (resistensi statis) \\
  C/C++ & 0.334 & Kuat pada KD-2 tetapi rentan pada KD-3 (lebih mudah di-\textit{reverse engineer}) \\
  Python & 0.296 & Skor rendah pada KD-2 (kontrol terbatas) dan KD-3 (\textit{signature} sudah familiar) \\
\end{longtable}

Berdasarkan hasil analisis MCDA-EWM, Rust dipilih sebagai bahasa implementasi untuk \textit{dropper spyware} mode \textit{stealth}. Keputusan ini didasarkan pada skor tertinggi yang diperoleh dan justifikasi kuat bahwa Rust secara efektif memadukan kontrol \textit{low-level} yang krusial untuk \textit{fileless execution} dengan resistensi inheren terhadap \textit{analisis statis} yang disasar oleh \textit{spyware} mode \textit{stealth} modern. Keputusan ini juga didukung oleh observasi industri bahwa \textit{threat actors} \textit{sophisticated} mulai mengadopsi Rust untuk pengembangan \textit{custom malware} dan \textit{reconnaissance tools}, menjadikan evaluasi \textit{dropper} berbasis Rust sangat relevan dan tepat waktu.
