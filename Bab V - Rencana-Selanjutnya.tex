% ==========================================
% BAB V RENCANA SELANJUTNYA
% ==========================================
\chapter{RENCANA SELANJUTNYA}
\label{chap:rencana-selanjutnya}

\section{Rencana Implementasi}
Subbab ini menjelaskan langkah implementasi prototipe dropper spyware mode stealth berbasis Rust, serta estimasi waktu dan sumber daya yang dibutuhkan.

\subsection{Langkah-langkah Implementasi}
Implementasi dibagi menjadi beberapa tahap utama yang mengikuti alur desain pada Bab IV, yaitu: pengembangan modul enkripsi payload, modul anti-analysis, modul eksekusi fileless, integrasi C2 dummy untuk pengujian, dan integrasi akhir beserta uji coba internal. Setiap tahap diakhiri dengan uji fungsional dasar untuk memastikan pemenuhan FR/NFR yang telah ditetapkan sebelumnya.

Tabel berikut merangkum timeline implementasi secara ringkas (estimasi berbasis minggu kalender):
{\centering
\begin{longtable}{p{0.8cm}p{3.8cm}p{6.2cm}p{2.2cm}}
  \caption{Timeline implementasi prototipe}\\
  \toprule
  \textbf{No} & \textbf{Tahap Implementasi} & \textbf{Aktivitas Kunci} & \textbf{Estimasi Waktu} \\
  \midrule
  \endfirsthead
  \toprule
  \textbf{No} & \textbf{Tahap Implementasi} & \textbf{Aktivitas Kunci} & \textbf{Estimasi Waktu} \\
  \midrule
  \endhead
  \bottomrule
  \endfoot
  \bottomrule
  \endlastfoot

  1 & Desain detail \& setup environment & Refinement spesifikasi teknis, setup repo, VM lab, toolchain Rust & 2 minggu \\
  2 & Modul Payload Encryptor \& Container & Implementasi AES, kompresi, embedding payload, uji enkripsi/dekripsi & 3 minggu \\
  3 & Modul Environment Checker & Implementasi anti-VM, anti-sandbox, anti-debug (CPUID, registry, process scan) & 3 minggu \\
  4 & Modul Fileless Loader \& Injection & Implementasi in-memory execution, process injection, uji di VM bersih & 4 minggu \\
  5 & Integrasi C2 Dummy \& Logging & Kanal C2 sederhana untuk verifikasi payload aktif, logging minimal & 2 minggu \\
  6 & Integrasi, hardening, dan uji internal & Penggabungan seluruh modul, optimasi, smoke test multi-VM & 2-3 minggu \\
\end{longtable}
\par}

\FloatBarrier

Perkiraan total durasi implementasi adalah sekitar 16-17 minggu, termasuk buffer untuk bug fixing ringan dan penyesuaian teknis.

\subsection{Alat yang Dibutuhkan}
Alat dan bahan yang dibutuhkan mencakup perangkat pengembangan, lingkungan virtual untuk pengujian aman, serta perangkat lunak pendukung analisis.

{\centering
\begin{longtable}{p{3cm}p{5cm}p{6cm}}
  \caption{Alat dan bahan pendukung implementasi}\\
  \toprule
  \textbf{Kategori} & \textbf{Alat/Bahan} & \textbf{Kegunaan} \\
  \midrule
  \endfirsthead
  \toprule
  \textbf{Kategori} & \textbf{Alat/Bahan} & \textbf{Kegunaan} \\
  \midrule
  \endhead
  \bottomrule
  \endfoot
  \bottomrule
  \endlastfoot

  Perangkat keras & Laptop/PC dengan minimal CPU 4 core, RAM 16 GB, SSD 512 GB & Pengembangan dan menjalankan beberapa VM uji \\
  Virtualisasi & VirtualBox/VMware/Hyper-V & Menyediakan VM Windows 10/11 untuk eksekusi dropper \\
  Bahasa \& toolchain & Rust (rustup, cargo, rust-analyzer) & Implementasi dropper dan modul pendukung \\
  Library utama & winapi, aes-gcm, sha2, sysinfo & Akses WinAPI, enkripsi, hashing, profiling lingkungan \\
  OS target & Windows 10/11 Pro (VM) & Lingkungan uji dropper dan anti-malware \\
  Anti-malware/EDR & Windows Defender, serta 2-3 produk komersial lain & Objek evaluasi kapabilitas deteksi \\
  Tool analisis & Sysmon, Procmon, Wireshark, x64dbg/WinDbg & Observasi perilaku, logging, dan debugging eksekusi \\
\end{longtable}
\par}

\FloatBarrier

Analisis biaya dapat disederhanakan dengan asumsi penggunaan lisensi akademik/gratis untuk sebagian besar perangkat lunak, sehingga biaya utama adalah penyediaan mesin pengembangan dan penyimpanan untuk log evaluasi.

\section{Rencana Evaluasi}
\subsection{Metode Pengujian}
Pengujian dilakukan di laboratorium terisolasi dengan beberapa VM Windows 10/11 yang masing-masing dipasangi produk anti-malware/EDR berbeda. Dropper dieksekusi dalam tiga skenario utama: pemindaian statis, eksekusi di VM bersih, dan eksekusi di VM yang dikonfigurasi menyerupai lingkungan analisis (VM/sandbox/debugger) untuk menguji mekanisme anti-analysis.

\subsection{Metrik Evaluasi}
Metrik digunakan untuk mengukur efektivitas dropper sekaligus kemampuan deteksi tiap produk anti-malware terhadap teknik evasi yang diimplementasikan. Ringkasan metrik disajikan pada Tabel berikut.

{\centering
\begin{longtable}{p{0.8cm}p{3cm}p{6cm}p{3cm}}
  \caption{Metrik evaluasi implementasi dropper}\\
  \toprule
  \textbf{No} & \textbf{Metrik} & \textbf{Definisi Singkat} & \textbf{Sumber Data} \\
  \midrule
  \endfirsthead
  \toprule
  \textbf{No} & \textbf{Metrik} & \textbf{Definisi Singkat} & \textbf{Sumber Data} \\
  \midrule
  \endhead
  \bottomrule
  \endfoot
  \bottomrule
  \endlastfoot

  1 & Detection Rate (DR) & Persentase engine/produk yang menandai dropper atau payload sebagai malicious pada pemindaian statis dan eksekusi & Log multi-engine scanner, log anti-malware di VM \\
  2 & Evasion Rate (ER) & Persentase eksekusi di mana dropper berhasil menjalankan payload tanpa terblokir pada tahap awal (before payload active) & Hasil eksekusi berulang pada tiap produk \\
  3 & Waktu Deteksi (Time-to-Detect) & Selisih waktu antara start dropper dan munculnya alert pertama dari anti-malware/EDR & Timestamp log produk keamanan dan Sysmon \\
  4 & Keberhasilan Anti-Analysis & Persentase eksekusi di VM/sandbox/debugger yang dihentikan oleh modul anti-VM/anti-debug sebelum payload aktif & Log internal dropper dan observasi perilaku di VM analisis \\
  5 & Stabilitas Eksekusi & Persentase run di VM bersih di mana dropper dan payload selesai tanpa crash atau error fatal & Catatan run dan log sistem di VM bersih \\
  6 & Jejak Forensik & Jumlah dan jenis artefak permanen (file, registry) yang tertinggal setelah eksekusi & Hasil inspeksi file system, registry, dan log Sysmon \\
\end{longtable}
\par}

\FloatBarrier

Nilai-nilai metrik ini kemudian dibandingkan dengan kriteria keberhasilan pada Subbab V.2.3 (misalnya batas maksimum DR statis, minimum ER, dan batas artefak yang diperbolehkan) untuk menilai apakah prototipe memenuhi tujuan penelitian.

\subsection{Kriteria Keberhasilan}
Kriteria keberhasilan dirumuskan berdasarkan CTQ yang telah didefinisikan di Bab III dan difokuskan pada efektivitas evasion serta stabilitas eksekusi.

{\centering
\begin{longtable}{p{0.8cm}p{3.5cm}p{6.5cm}p{2.4cm}}
  \caption{Kriteria keberhasilan prototipe dropper}\\
  \toprule
  \textbf{No} & \textbf{Kriteria} & \textbf{Deskripsi Singkat} & \textbf{Target} \\
  \midrule
  \endfirsthead
  \toprule
  \textbf{No} & \textbf{Kriteria} & \textbf{Deskripsi Singkat} & \textbf{Target} \\
  \midrule
  \endhead
  \bottomrule
  \endfoot
  \bottomrule
  \endlastfoot

  1 & CTQ-01 - Payload concealment & Payload terenkripsi tidak terdeteksi pada fase pemindaian statis di mayoritas engine & Detection rate statis $\leq 30\%$ \\
  2 & CTQ-02 - Signature evasion & Biner dropper lolos dari signature-based detection pada beberapa produk utama & Minimal 75\% produk komersial tidak mendeteksi pada eksekusi awal \\
  3 & CTQ-03 - Environment \& behavioral stealth & Mekanisme anti-VM/anti-debug mencegah eksekusi penuh di VM analisis, serta eksekusi di VM bersih tidak langsung memicu alert perilaku & $> 80\%$ run di VM analisis berhenti sebelum payload aktif, dan minimal satu produk terlambat mendeteksi pada VM bersih \\
  4 & Stabilitas eksekusi & Dropper berhasil mengeksekusi payload tanpa crash pada VM bersih & Keberhasilan eksekusi $\geq 95\%$ dari total percobaan \\
  5 & Minimnya artefak forensik & Tidak ada file permanen atau entri registry baru terkait dropper setelah eksekusi & Tidak ditemukan artefak disk yang persisten pada inspeksi manual \\
\end{longtable}
\par}

\FloatBarrier

Jika sebagian kriteria tidak tercapai, hasil tersebut tetap didokumentasikan sebagai temuan ilmiah dan dasar rekomendasi perbaikan deteksi maupun desain dropper pada penelitian lanjutan.

\section{Analisis Risiko}
Analisis risiko disusun secara ringkas untuk mengidentifikasi hambatan utama selama implementasi dan evaluasi prototipe, serta strategi mitigasinya.

{\centering
\begin{longtable}{p{0.8cm}p{4cm}p{4.5cm}p{5cm}}
  \caption{Risiko implementasi dan mitigasi}\\
  \toprule
  \textbf{No} & \textbf{Risiko} & \textbf{Dampak Utama} & \textbf{Mitigasi Singkat} \\
  \midrule
  \endfirsthead
  \toprule
  \textbf{No} & \textbf{Risiko} & \textbf{Dampak Utama} & \textbf{Mitigasi Singkat} \\
  \midrule
  \endhead
  \bottomrule
  \endfoot
  \bottomrule
  \endlastfoot

  1 & Implementasi fileless injection lebih sulit dari perkiraan & Timeline pengembangan mundur, beberapa FR tidak tercapai & Memulai lebih awal pada modul loader, membatasi teknik injection ke 1-2 metode yang paling stabil, dan menyiapkan fallback berupa eksekusi semi-fileless jika perlu \\
  2 & Deteksi anti-malware terlalu tinggi sehingga tidak merepresentasikan dropper stealth & Evaluasi tidak menggambarkan gap deteksi yang realistis & Iteratif menyesuaikan konfigurasi enkripsi, packing, dan anti-analysis sampai biner mencapai profil deteksi yang seimbang namun tetap aman didokumentasikan \\
  3 & Lab tidak sepenuhnya terisolasi & Potensi gangguan ke jaringan luar, risiko etis & Menggunakan jaringan host-only/NAT tertutup, memblokir akses internet dari VM, dan menggunakan C2 dummy lokal saja \\
  4 & Keterbatasan waktu dibanding kompleksitas eksperimen & Beberapa skenario uji tidak sempat dijalankan & Memprioritaskan produk anti-malware yang paling relevan, menyusun jadwal pengujian terstruktur, dan menyimpan skenario tambahan sebagai pekerjaan lanjutan \\
  5 & Koordinasi dengan pembimbing tidak lancar & Revisi konsep terlambat, kualitas artefak menurun & Menetapkan jadwal konsultasi rutin dan mengirimkan progres singkat tertulis agar umpan balik tetap terjaga meski jadwal padat \\
\end{longtable}
\par}

\FloatBarrier
Melalui perencanaan implementasi yang terstruktur dan evaluasi yang ketat, penelitian ini diharapkan mampu memberikan data empiris mengenai celah keamanan pada sistem deteksi modern terhadap ancaman \textit{dropper} berbasis Rust.
